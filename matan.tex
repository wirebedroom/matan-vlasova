\documentclass[a4paper,12pt]{article} % добавить leqno в [] для нумерации слева

%%% Работа с русским языком
\usepackage{cmap}					% поиск в PDF
\usepackage{mathtext} 				% русские буквы в фомулах
\usepackage[T2A]{fontenc}			% кодировка
\usepackage[utf8]{inputenc}			% кодировка исходного текста
\usepackage[english,russian]{babel}	% локализация и переносы

%%% Дополнительная работа с математикой
\usepackage{amsmath,amsfonts,amssymb,amsthm,mathtools} % AMS
\usepackage{icomma} % "Умная" запятая: $0,2$ --- число, $0, 2$ --- перечисление

%% Номера формул
\mathtoolsset{showonlyrefs=true} % Показывать номера только у тех формул, на которые есть \eqref{} в тексте.

%% Шрифты
\usepackage{euscript}	 % Шрифт Евклид
\usepackage{mathrsfs} % Красивый матшрифт
%% Акценты
\usepackage{accents}

%% Свои команды
\DeclareMathOperator{\sgn}{\mathop{sgn}}
% Теоремы
\newtheorem{definition}{Определение}[subsection]
\newtheorem{theorem}{Теорема}[subsection]
\newtheorem{corollary}{Corollary}[theorem]
\newtheorem{lemma}{Утверждение}[subsection]
\theoremstyle{remark}
\newtheorem*{remark}{Замечание}
%\newtheoremstyle{named}{}{}{\itshape}{}{\bfseries}{.}{.5em}{\thmnote{#3's }#1}
%\theoremstyle{named}
%\newtheorem*{namedtheorem}{Theorem}

%% Colors
\usepackage{xcolor}
\definecolor{nord9}{RGB}{129,161,193}
\definecolor{nord8}{RGB}{136,192,208}

\usepackage{textcomp}

%% TikZ
\usepackage{tikz}
\usetikzlibrary{arrows.meta}
\usetikzlibrary{shapes.geometric,positioning}
%\usetikzlibrary{patterns.meta}

% figure support
\usepackage{import}
\usepackage{pdfpages}
\usepackage{transparent}

\newcommand{\incfig}[2][1]{%
    \def\svgwidth{#1\columnwidth}
    \import{./figures/}{#2.pdf_tex}
}

\pdfsuppresswarningpagegroup=1



%% Перенос знаков в формулах (по Львовскому)
\newcommand*{\hm}[1]{#1\nobreak\discretionary{}
{\hbox{$\mathsurround=0pt #1$}}{}}

%%% Заголовок
\author{Власова Елена Александровна}
\title{Лекции по математическому анализу для 1 курса ФН2, 3}
\date{2024-2025 год.}

\begin{document} % конец преамбулы, начало документа

\maketitle

\newpage
\tableofcontents
\newpage

\part*{Элементарные функции и их пределы}

\section{Введение}
\subsection{Элементы теории множеств}
\subsection{Кванторные операции}
\subsection{Метод математической индукции}

\section{Множество действительных чисел}
\subsection{Аксиоматика действительных чисел}
\begin{definition}
	Множество $\mathbb{R}$ называется множеством действительных чисел, если элементы этого множества удовлетворяют следующему комплексу условий:
\begin{enumerate}
	\item На множестве $\mathbb{R}$ определена операция сложения ``+'', то есть задано отображение, которое каждой упорядоченной паре $(x, y)\in \mathbb{R}^2$ ставит в соответствие элемент из $\mathbb{R}$, называемый суммой $x+y$ и удовлетворяющий следующим аксиомам:
		\begin{enumerate}
			\item $\exists 0\in \mathbb{R}$, такой, что $\forall x\in \mathbb{R}:x+0=0+x=x$;
			\item $\forall x \ \exists$ противоположный элемент ``$-x$'', такой, что $x+(-x)=(-x)+x=0$;
			\item Ассоциативность. $\forall x, y, z\in \mathbb{R}:(x+y)+z=x+(y+z)$;
			\item Коммутативность. $\forall x, y\in \mathbb{R}:x+y=y+x$.
		\end{enumerate}
	\item На $\mathbb{R}$ определена операция умножения ``$\cdot$'', то есть $\forall(x, y)\in\mathbb{R}^2$ ставится в соответствие элемент $(x\cdot y)\in \mathbb{R}$.
		\begin{enumerate}
			\item $\exists $ нейтральный элемент $1\in \mathbb{R}$, такой, что $\forall x\in \mathbb{R}:1\cdot x=x\cdot 1=x$;
			\item $\forall x\in \mathbb{R}\backslash\{0\} \ \exists $ обратный элемент ``$x^{-1}$'', такой, что $x\cdot x^{-1}=x^{-1}\cdot x=1$;
			\item Ассоциативность. $\forall x, y, z\in \mathbb{R}\backslash\{0\}:(x\cdot y)\cdot z=x\cdot (y\cdot z)$;
			\item Коммутативность. $\forall x, y\in \mathbb{R}\backslash\{0\}:x\cdot y=y\cdot x$.
		\end{enumerate}
\end{enumerate}
	Операция умножения дистрибутивна по отношению к сложению. 
		$$\forall x, y, z\in \mathbb{R}:(x+y)z=xz+yz$$
\begin{enumerate}
\setcounter{enumi}{2}
	\item Отношения порядка. Для $\mathbb{R}$ определено отношение ``$\le $ ''.
		\begin{enumerate}
			\item $\forall x\in \mathbb{R}:x\le x$;
			\item $\forall x, y\in \mathbb{R}:(x\le y \land y\le x)\implies x=y$;
				
				\ldots
		\end{enumerate}
\end{enumerate}
	
\end{definition}

\subsection{Геометрическая интерпретация $\mathbb{R}$}
\subsection{Числовые промежутки}
\subsection{Бесконечные числовые промежутки}
\subsection{Окрестности точки}
\subsection{Принцип вложенных отрезков (Коши-Кантора)}
\begin{definition}
	Пусть $\{x_n\}_{n=1}^{\infty}$ --- последовательность некоторых множеств. Если $\forall n\in \mathbb{N}:X_n\supset X_{n+1}$, то эта последовательность называется последовательностью вложенных отрезков.
\end{definition}

\subsection{Ограниченные и неограниченные числовые множества}
\subsection{Точные грани числового множества}
\subsection{Принцип Архимеда}


\section{Функции или отображения}
\subsection{Понятие функции}
\subsection{Ограниченные и неограниченные числовые множества}
\subsection{Обратные функции}
\subsection{Чётные и нечётные функции}
\subsection{Периодические функции}
\subsection{Сложная функция (композиция)}
\subsection{Основные элементарные функции}

\section{Числовые последовательности и их пределы}



\begin{definition}
	$f:\mathbb{N}\rightarrow \mathbb{R}$ --- числовая последовательность, т.е. $\{x_n\}_{n=1}^\infty$, $x_n \in \mathbb{R}$.
\end{definition}



\subsection{Ограниченные и неограниченные числовые последовательности}

\begin{definition}
	Числовая последовательность $\{x_n\}_{n=1}^\infty$ называется 

	\begin{enumerate}
		\item ограниченной сверху, если $\exists M\in \mathbb{R}: \forall n\in \mathbb{N}: \ x_n\le M$;
		\item ограниченной снизу, если \space $\exists M\in \mathbb{R}: \forall n\in \mathbb{N}: \ x_n\ge M$;
		\item ограниченной, если \space\space\space\space\space\space\space\space\space $\exists M\in \mathbb{R}: \forall n\in \mathbb{N}:|x_n| \le M$;
		\item неограниченной, если \space\space\space\space\space\space $\exists M\in \mathbb{R}: \forall n\in \mathbb{N}:|x_n|> M$;
	\end{enumerate}
\end{definition}



\subsection{Предел числовой последовательности}
\begin{definition}
	Число $a\in\mathbb{R}$ называется пределом числовой последовательности, если $\forall \varepsilon>0$ существует такой номер $n$, зависящий от $\varepsilon$, что $\forall$ натурального числа $N>n$ верно неравенство $|x_n - a| < \varepsilon$.
	$$\lim_{n\to\infty} x_n = a$$
\end{definition}



\subsection{Бесконечные пределы}



\subsection{Свойства сходящихся последовательностей}
\begin{theorem}[о единственности предела]
	Любая сходящаяся последовательность имеет только один предел.	
\end{theorem}
\begin{proof}
	"От противного". Пусть $\{x_n\}_{n=1}^{\infty}$ --- сходящаяся последовательность. Предположим, что $\exists \lim_{n \to \infty} x_{n} = a$ и $\exists \lim_{n \to \infty} x_{n} = b$, причем $a\neq b$. Пусть для определенности $a<b$.
	\[
	\lim_{n \to \infty} x_{n} = a \iff \forall \varepsilon>0 \quad \exists N_1=N_1(\varepsilon)\in \mathbb{N} : \forall n>N_1 : |x_{n}-a|<\varepsilon
	,\] 
	\[
	\lim_{n \to \infty} x_{n} = b \iff \forall \varepsilon>0 \quad \exists N_2=N_2(\varepsilon)\in \mathbb{N} : \forall n>N_2 : |x_{n}-b|<\varepsilon
	.\] 
	\[
		N = \max\{N_1, n_2\} \implies \forall n>N : 
		\begin{cases}
			|x_{n}-a| < \varepsilon, \\
			|x_{n}-b| < \varepsilon.
		\end{cases}
	\] 
\begin{center}
	\begin{tikzpicture}
		% Числовая прямая x.
		\draw[-{Classical TikZ Rightarrow[length=1mm]}] (0,0) -- (12,0);	
		\draw (12,0) node[anchor=north] (x) {$x$};
		% Точки
		\filldraw[black] (3,0) circle (0.05) node[anchor=north] (a) {$a$}; 
		\filldraw[black] (9,0) circle (0.05) node[anchor=north] (b) {$b$}; 
		\draw (1,0) node[anchor=north] (a-e) {$a-\varepsilon$};
		\draw (5,0) node[anchor=north] (a+e) {$a+\varepsilon$};
		\draw (7,0) node[anchor=north] (b-e) {$b-\varepsilon$};
		\draw (11,0) node[anchor=north] (b+e) {$b+\varepsilon$};
		\draw (6,2) node[anchor=north, yshift=0cm] (xn) {$\forall n>N:x_{n}$};
		% Интервалы
		\draw (1,0) to[out=70,in=110] (5,0);
		\draw (7,0) to[out=70,in=110] (11,0);
		% Стрелки
		\draw[->] (6.8,1.4) to (3.6,0.5);
		\draw[->] (6.8,1.4) to (8.4,0.5);
	\end{tikzpicture}
\end{center}
Выберем $\varepsilon = \frac{b-a}{4} > 0$. Найдем $N_1(\varepsilon), N_2(\varepsilon), N=\max\{N_1, N_2\}$, тогда
\[
	\forall n>N \quad |x_{n}-a| < \frac{b-a}{4}, \quad |x_{n}-b| < \frac{b-a}{4}
.\] 
Следовательно,
\[
	0 < b-a = |b-a| = |b-x_{n}+x_{n}-a| \le |x_{n}-b|+|x_{n}-a| < \frac{b-a}{2}
,\] 
то есть
\[
0 < b-a < \frac{b-a}{2}
.\] 
Мы пришли к противоречию, следовательно, $a=b \implies \{x_n\}_{n=1}^{\infty}$ имеет единственный предел.
\end{proof}

\begin{theorem}[об ограниченности сходящейся последовательности]
	Любая сходящаяся последовательность является ограниченной.	
	\begin{center}
		\begin{tikzpicture}
			% Числовая прямая x.
				\draw[-{Classical TikZ Rightarrow[length=1mm]}] (0,0) -- (12,0);        
			% Точки
				\filldraw[black] (4,0) circle (0.05) node[anchor=north] (x_1) {$x_1$}; 
				\filldraw[black] (10.3,0) circle (0.05) node[anchor=north] (x_2) {$x_2$}; 
				\filldraw[black] (5.3,0) circle (0.05) node[anchor=north] (x_3) {$x_3$}; 
				\filldraw[black] (7.8,0) circle (0.05) node[anchor=north] (x_n) {$x_n$}; 
				\draw[shift={(6,0)},color=black] (0pt,3pt) -- (0pt,-3pt) node[anchor=north] (a) {$a$};
				\draw (3,0) node[anchor=north] (a-e) {$a-\varepsilon$};
				\draw (9,0) node[anchor=north] (a+e) {$a+\varepsilon$};
			% Окрестность
				\draw (8,2) node[anchor=north] (Ua) {$U(a)$};
			% Интервалы 
				\draw (3,0) to[out=70,in=110] (9,0);
			% Отрезок
				\draw (1,0) -- ++(0,2) -- ++(10,0) -- ++(0,-2);
		\end{tikzpicture}
	\end{center}
\end{theorem}
\begin{proof}
	Если $\{x_n\}_{n=1}^{\infty}$ сходится, то 
	\[
	\exists \lim_{n \to \infty} = a\in \mathbb{R} \implies \forall \varepsilon>0 \quad \exists N=N(\varepsilon)\in \mathbb{N} \quad \forall n>N : |x_{n}-a|<\varepsilon
	\]
	Пусть $\varepsilon = 1 \implies \exists N=N(1) \quad \forall n>N : |x_{n}-a| < 1$. Следовательно,
	\[
	|x_{n}| = |x_{n}-a+a| \le |x_{n}-a|+|a| < 1+|a|
	.\] 
	Пусть $M_0=1+|a| \implies \forall n>N : x_{n} < M_0$. 
	
	\noindent Пусть $M=\max\{|x_1|, |x_2|, \ldots, |x_{n}|, M_0\}$, тогда $\forall n\in \mathbb{N} : x_{n}\le M\implies \{x_n\}_{n=1}^{\infty}$ является ограниченной.
\end{proof}

\begin{remark}
	Ограниченность является необходимым условием сходимости числовой последовательности. В то же время условие ограниченности не является достаточным для сходимости числовой последовательности.
	Например, $\{(-1)^n\}_{n=1}^{\infty}$ --- ограниченная, но не сходящаяся числовая последовательность.
\end{remark}



\subsection{Монотонные числовые последовательности}
\begin{definition}
	Числовая последовательность $\{x_n\}_{n=1}^{\infty}$ называется 
	\begin{enumerate}
		\item возрастающей, если \ \ $\forall n\in \mathbb{N}:x_{n}<x_{n+1}$;
		\item убывающей, если \ \ \ \ \ \ \ $\forall n\in \mathbb{N}:x_{n}>x_{n+1}$;
		\item неубывающей, если \ \ \ \ $\forall n\in \mathbb{N}:x_{n}\le  x_{n+1}$;
		\item невозрастающей, если $\forall n\in \mathbb{N}:x_{n}\ge x_{n+1}$
	\end{enumerate}
\end{definition}
Для монотонных числовых последовательностей ограниченность является достаточным условием для сходимости.
\begin{theorem}[Вейерштрасса о сходимости монотонных числовых последовательностей]
	Если последовательность не убывает и ограничена сверху, то она является сходящейся.
	Если последовательность не возрастает и ограничена снизу, то она является сходящейся. В общем, любая монотонная последовательность сходится.
\end{theorem}
\begin{proof}
	Пусть $\{x_n\}_{n=1}^{\infty}$ не убывает и ограничена сверху $\implies$ \\
	$\implies \exists M\in \mathbb{R} :\forall n\in \mathbb{N}:x_{n}\le M\implies$ \\
	\indent $\implies$ множество значений этой последовательности \\
	\indent \indent \ \ \ $\{x_1, x_2,\ldots, x_{n}, \ldots\}=A$ является ограниченным \\ 
	\indent \indent \ \ \ сверху числовым множеством $\implies$ \\
	\indent \indent \indent \indent \indent \indent $\implies \exists \sup A\in \{x_n\}_{n=1}^{\infty}=a$, то есть
	\begin{enumerate}
		\item $\forall n\in \mathbb{N}:x_{n}\le a$;
		\item $\forall \varepsilon>0 \ \exists N=N(\varepsilon)\in \mathbb{N}:x_{N}>a-\varepsilon$.
	\end{enumerate}

\begin{center}
	\begin{tikzpicture}
		\draw[-{Classical TikZ Rightarrow[length=1mm]}] (0,0) -- (12,0);	
		% Числовая прямая
		\filldraw[black] (4.5,0) circle (0.05) node[anchor=south]{$x_N$}; %node[yshift=-1.4mm]
		% Точки
		%\draw[shift={(5,0)},color=black] (0pt,3pt) -- (0pt,-3pt) node[anchor=north] (a) {$a$};
		\draw (6,0) node[anchor=north] (a) {$a$};
		\draw (2,0) node[anchor=north] (a-ep) {$a-\varepsilon$};
		\draw (10,0) node[anchor=north] (a+ep){$a+\varepsilon$};
		%\draw[shift={(2,0)},color=black] (0pt,3pt) -- (0pt,-3pt) node[anchor=north] (a-ep) {$a-\varepsilon$};
		%\draw[shift={(8,0)},color=black] (0pt,3pt) -- (0pt,-3pt) node[anchor=north] (a+ep){$a+\varepsilon$};
		% Интервалы
		\draw (2,0) to[out=70,in=110] (6,0);
		\draw (6,0) to[out=70,in=110] (10,0);
	\end{tikzpicture}
\end{center}
$\{x_n\}_{n=1}^{\infty}$ --- неубывающая последовательность, то есть 
\begin{multline}
\forall n>N=N(\varepsilon):x_{n}\ge x_N \implies \\
\implies a-\varepsilon<x_N\le x_{n}\le a<a+\varepsilon\implies \\
\implies a-\varepsilon<x_{n}<a+\varepsilon \implies |x_{n}-a|<\varepsilon \implies \\
\implies \forall \varepsilon>0 \quad \exists N=N(\varepsilon)\in \mathbb{N}:\forall n>N:|x_{n}-a|<\varepsilon \implies \\
\implies \exists \lim_{n \to \infty}x_{n} = a\in \mathbb{R} \implies \{x_n\}_{n=1}^{\infty} \text{ сходится.}
\end{multline}
Если $\{x_n\}_{n=1}^{\infty}$ --- невозрастающая и ограниченная снизу последовательность, то 
\[
\exists \lim_{n \to \infty} x_{n}=\inf A, A = \{x_1, x_2, \ldots, x_{n}, \ldots\} 
.\] 
Доказательство аналогично.
\end{proof}

\subsection{Число $e$}

\subsection{Гиперболические функции}
\subsection{Предельные точки числового множества}
\begin{definition}
	Точка $a\in \mathbb{R}$ называется предельной точкой множества $X\subset \mathbb{R}$, если любая окрестность $U(a)$ содержит бесконечно много элементов множества $X$.
\end{definition}
\begin{remark}
	Множество $A$ называется бесконечным или содержащим бесконечно много элементов, если при вычитании из $A$ любого его конечного подмножества получается непустое множество.
\end{remark}

Множество всех предельных точек множества $X$ называется производным множеством для $X$ и обозначается $X'$.

\begin{lemma}
	Точка $a\in \mathbb{R}$ является предельной для $X\subset \mathbb{R}\iff$ в любой проколотой $\delta$-окрестности точки $a$ содержится хотя бы один элемент множества $X$, т.е.
	\[
		\forall \delta>0 \quad \exists x\in X \cap \accentset{\circ}{U}(a)
	.\]
\end{lemma}
\begin{proof}
	$(\implies)$ Необходимость.
	
	$a$ --- предельная для $X\subset \mathbb{R}\implies$ \\ 
	\indent $\implies$ любая $U(a)$ содержит бесконечно много элементов из $X \implies$ \\ 
	\indent \indent $\implies \accentset{\circ}{U}(a)$ тоже содержит бесконечно много элементов из $X\implies$ \\ 
	\indent \indent \indent $\implies$ любая $\accentset{\circ}{U}$ содержит хотя бы один элемент $x\in X$.

	\noindent $(\impliedby)$ Достаточность.
	\[
		\forall \delta>0 \quad \exists x\in X \cap \accentset{\circ}{U}(a)
	.\]
	Выберем любую $U(a)$. Тогда
\[
\exists \delta_1>0 : \accentset{\circ}{U}(a)\subset U(a) \implies \exists x_1\in X : x_1\in \accentset{\circ}{U}_{\delta_1}(a)
.\] 
\begin{center}
	\begin{tikzpicture}
		\draw[-{Classical TikZ Rightarrow[length=1mm]}] (0,0) -- (12,0);	
		% Числовая прямая
		\filldraw[black] (3.7,0) circle (0.05) node[anchor=south]{$x_1$};
		% Точки
		\draw (6,0) node[anchor=north] (a)   {$a$};
		\draw (1,0) node[anchor=north] (b)   {$b$};
		\draw (11,0) node[anchor=north] (c)   {$c$};
		\draw (3,0) node[anchor=north] (a-d) {$a-\delta_1$};
		\draw (9,0) node[anchor=north] (a+d) {$a+\delta_1$};
		\draw (7,0) node[anchor=north, yshift=3.4cm] (Ua) {$U(a)$};
		\draw (5,0) node[anchor=north, yshift=1.6cm] (Uo) {$\accentset{\circ}{U}_{\delta_1}(a)$};
		% Интервалы
		\draw (1,0) to[out=70,in=110] (11,0);
		\draw (3,0) to[out=70,in=110] (6,0);
		\draw (6,0) to[out=70,in=110] (9,0);
	\end{tikzpicture}
\end{center}
Пусть $\delta_2 = \frac{|x_1-a|}{2} > 0$. Тогда
\[
	\exists x_2 \in \accentset{\circ}{U}_{\delta_2}(a) : x_2\neq x_1
.\] 

\begin{center}
	\begin{tikzpicture}
		% Числовая прямая
		\draw[-{Classical TikZ Rightarrow[length=1mm]}] (0,0) -- (12,0);	
		% Точки
		\filldraw[black] (2.1666,0) circle (0.05) node[anchor=south]{$x_1$}; 
		\filldraw[black] (5,0) circle (0.05) node[anchor=south]{$x_2$}; 
		\draw (6,0) node[anchor=north] (a)   {$a$};
		\draw (1,0) node[anchor=north] (a-d) {$a-\delta_1$};
		\draw (11,0) node[anchor=north] (a+d) {$a+\delta_1$};
		\draw (3.8334,0) node[anchor=north] (a-d) {$a-\delta_2$};
		\draw (8.1666,0) node[anchor=north] (a+d) {$a+\delta_2$};
		\draw (9.5,0) node[anchor=north, yshift=2.1cm] (Uo_1) {$\accentset{\circ}{U}_{\delta_1}(a)$};
		\draw (8.1,0) node[anchor=north, yshift=1.25cm] (Uo_2) {$\accentset{\circ}{U}_{\delta_2}(a)$};
		% Интервалы
		\draw (1,0) to[out=80,in=100] (6,0);
		\draw (6,0) to[out=80,in=100] (11,0);
		\draw (3.8334,0) to[out=70,in=110] (6,0);
		\draw (6,0) to[out=70,in=110] (8.1666,0);
	\end{tikzpicture}
\end{center}
Пусть $\delta_3 = \frac{|x_2-a|}{2} > 0$. Тогда 
\[
	\exists x_3 \in \accentset{\circ}{U}_{\delta_3}(a) : x_3\neq x_2 
\] 
и т.д. На шаге $n$:
\[
\delta_n = \frac{|x_{n-1}-a|}{2} > 0 \implies \exists x_n \in \accentset{\circ}{U}_{\delta_n}(a) : x_n\neq x_{k}, k =1, 2, \ldots, n-1 
.\] 
Таким образом,
\[
	\exists \{x_n\}_{n=1}^{\infty}\in U(a) : x_{n}\in X, x_{n}\neq x_k, n\neq k,
\] 
а значит, любая окрестность $U(a)$ содержит бесконечно много элементов из $X \implies a$ --- предельная точка.
\end{proof}

\begin{lemma}
	Если точка $a\in \mathbb{R}$ является предельной точкой для множества $X\subset \mathbb{R}$, то
	\[
	\exists \{x_n\}_{n=1}^{\infty}\subset X : \lim_{n \to \infty} x_n = a
	.\] 
\end{lemma}

\begin{proof}
	$a$ --- предельная точка для $X\subset \mathbb{R} \iff \forall \delta>0 \quad \accentset{\circ}{U}_\delta(a)$ содержит хотя бы одну точку множества $X$ (по утверждению 1).
	$\newline$
	Выберем $\{\delta_n\}_{n=1}^{\infty}, \delta_n=\frac{1}{n}>0$, тогда
	\[
		\forall n\in \mathbb{N} \quad \exists x_n\in X : x_{n}\in \accentset{\circ}{U}_{\delta_n}(a)
	,\] 
	то есть 
	\[
	0 < |x_n-a| < \frac{1}{n}
	.\] 
Т.к. $\lim_{n \to \infty} \frac{1}{n} = 0$, 
\[
	\forall \varepsilon>0 \quad \exists N=N(\varepsilon)\in \mathbb{N} \quad \forall n>N : \frac{1}{n} < \varepsilon
,\] 
а значит,
\[
|x_{n}-a| < \frac{1}{n} < \varepsilon \implies \lim_{n \to \infty} x_{n} = a
.\] 
\end{proof}

\begin{theorem}[принцип Больцано-Вейерштрасса]
	Любое ограниченное бесконечное числовое множество имеет хотя бы одну предельную точку.
\end{theorem}
\begin{proof}
	Пусть $X$ --- бесконечное ограниченное множество, то есть $\exists I_1 =[a_1, b_1] : X\subset [a_1, b_1]$.
	Пусть $c_1=\frac{a_1+b_1}{2}$, т.е. середина отрезка $I_1$.
	\begin{center}
		\begin{tikzpicture}
			%\fill[nord8] (1,0)rectangle(11,1); 
			%% Числовая прямая x.
			\draw[-{Classical TikZ Rightarrow[length=1mm]}] (0,0) -- (12,0);	
			\draw (12,0) node[anchor=north] (x) {$x$};
			%% a1, b1, c1
			\filldraw[black] (1,0) circle (0.05) node[anchor=north] (a1) {$a_1$};
			\filldraw[black] (11,0) circle (0.05) node[anchor=north] (b1) {$b_1$};
			\draw[shift={(6,0)},color=black] (0pt,3pt) -- (0pt,-3pt) node[anchor=north] (c1) {$c_1$};
			\draw (6,0) node[anchor=north, yshift=1.7cm] (I_1) {$I_1$};
			%% X and arrow
			\draw (4,-1) node[anchor=north] (X) {$X$};
			\draw[->] (X) to[out=135,in=225] (4,0.5);
			%% Отрезок
			\draw (1,0) -- ++(0,1) -- ++(10,0) -- ++(0,-1);
		\end{tikzpicture}
	\end{center}
	Так как множество $X$ бесконечное, то либо отрезок $[a_1, c_1]$, либо отрезок $[c_1, b_1]$ содержит бесконечно много элементов множества $X$. Обозначим ту половину отрезка $I_1$, которая содержит бесконечно много элементов множества $X$ через $I_2 = [a_2, b_2], I_2\subset I_1$. Выразим длину отрезка $I_2$:
\[
	|I_2| = b_2-a_2 = \frac{b_1-a_1}{2} = \frac{|I_1|}{2}
.\] 
На отрезке $I_2$ содержится бесконечно много элементов множества $X$.
Пусть $c_2=\frac{a_2+b_2}{2}$ --- середина $I_2$, тогда либо $[a_2, c_2]$, либо $[c_2, b_2]$ содержит бесконечно много элементов множества $X$. Обозначим ту половину $I_2$, где бесконечно много элементов множества $X$ через $I_3 = [a_3, b_3]$. Тогда
\[
|I_3| = \frac{|I_1|}{2^2}
\] 
и т.д. На шаге n: $I_n=[a_n, b_n], c_n = \frac{a_n + b_n}{2}$ --- середина $I_n$, $I_n$ содержит бесконечно много элементов из $X$, тогда либо $[a_n, c_n]$, либо $[c_n, b_n]$ содержит бесконечно много элементов из $X \implies I_{n+1}=[a_{n+1}, b_{n+1}]\subset I_n$ и содержит бесконечно много элементов из $X$. Таким образом, мы получили последовательность вложенных отрезков $\{I_n\}_{n=1}^{\infty} : I_1\supset I_2\supset \ldots\supset I_n\supset I_{n+1}\supset \ldots$
\begin{multline}
	|I_n|=\frac{|I_1|}{2^{n-1}} \implies \lim_{n \to \infty} \frac{|I_1|}{2^{n-1}} = 0 \implies \\
	\implies \forall \varepsilon>0 \quad \exists N=N(\varepsilon)\in \mathbb{N} \quad \forall n>N : |I_n| < \varepsilon.
\end{multline}
По принципу Коши-Кантора $\exists !$ общая точка $c$, т.е. $\forall n\in \mathbb{N} : c\in I_n$.
\[
	\forall U(c) \quad \exists \varepsilon>0 \quad U_\varepsilon(c) \subset U(c) \implies \exists n\in \mathbb{N} : I_n=[a_n, b_n] \subset U_\varepsilon(c)
\] 
(например, $|I_n| < \frac{\varepsilon}{2}$).

\begin{center}
	\begin{tikzpicture}
		% Числовая прямая
		\draw[-{Classical TikZ Rightarrow[length=1mm]}] (0,0) -- (12,0);	
		% Точки
		\draw[shift={(6,0)},  color=black] (0pt,3pt) -- (0pt,-3pt) node[anchor=north] (c)   {$c$};
		\filldraw[black] (4.5,0) circle (0.05) node[anchor=north] {$a_n$}; 
		\filldraw[black] (7.5,0) circle (0.05) node[anchor=north] {$b_n$}; 
		%%\draw[shift={(4.5,0)},color=black] (0pt,3pt) -- (0pt,-3pt) node[anchor=north] (a_n) {$a$};
		%%\draw[shift={(7.5,0)},color=black] (0pt,3pt) -- (0pt,-3pt) node[anchor=north] (b_n) {$b$};
		\draw (3,0) node[anchor=north] (a-d) {$c-\varepsilon$};
		\draw (9,0) node[anchor=north] (a+d) {$c+\varepsilon$};
		\draw (7,0) node[anchor=north, yshift=3.4cm] (Uc) {$U(c)$};
		% Интервалы
		\draw (1,0) to[out=70,in=110] (11,0);
		\draw (3,0) to[out=70,in=110] (9,0);
		\draw (4.5,0) -- ++(0,0.7) -- ++(3,0) -- ++(0,-0.7);
	\end{tikzpicture}
\end{center}
Отрезок $I_n$ содержит бесконечно много элементов множества $X$ по построению последовательности $\{I_n\}_{n=1}^{\infty} \implies$ окрестность $U(c)$ содержит бесконечно много элементов из $X \implies c$ --- предельная.
\end{proof}

\subsection{Предельные точки числовых последовательностей}

\end{document} % конец документа
