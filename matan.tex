\documentclass[a4paper,12pt]{article} % добавить leqno в [] для нумерации слева

%%% Работа с русским языком
\usepackage{cmap}					% поиск в PDF
\usepackage{mathtext} 				% русские буквы в фомулах
\usepackage[T2A]{fontenc}			% кодировка
\usepackage[utf8]{inputenc}			% кодировка исходного текста
\usepackage[english,russian]{babel}	% локализация и переносы

%%% Дополнительная работа с математикой
\usepackage{amsmath,amsfonts,amssymb,amsthm,mathtools} % AMS
\usepackage{icomma} % "Умная" запятая: $0,2$ --- число, $0, 2$ --- перечисление

%% Номера формул
\mathtoolsset{showonlyrefs=true} % Показывать номера только у тех формул, на которые есть \eqref{} в тексте.

%% Шрифты
\usepackage{euscript}	 % Шрифт Евклид
\usepackage{mathrsfs} % Красивый матшрифт

%% Свои команды
\DeclareMathOperator{\sgn}{\mathop{sgn}}
\newtheorem{definition}{Определение}
\newtheorem{theorem}{Теорема}
\newtheorem{lemma}{Утверждение}

%% Перенос знаков в формулах (по Львовскому)
\newcommand*{\hm}[1]{#1\nobreak\discretionary{}
{\hbox{$\mathsurround=0pt #1$}}{}}

%%% Заголовок
\author{Власова Елена Александровна}
\title{Лекции по математическому анализу для 1 курса ФН2, 3}
\date{2024-2025 год.}

\begin{document} % конец преамбулы, начало документа

\maketitle

\newpage
\tableofcontents
\newpage

\part*{Элементарные функции и их пределы}

\section{Введение}
\subsection{Элементы теории множеств}
\subsection{Кванторные операции}
\subsection{Метод математической индукции}

\section{Множество действительных чисел}
\subsection{Аксиоматика действительных чисел}
\begin{definition}
	Множество $\mathbb{R}$ называется множеством действительных чисел, если элементы этого множества удовлетворяют следующему комплексу условий:
\begin{enumerate}
	\item На множестве $\mathbb{R}$ определена операция сложения ``+'', то есть задано отображение, которое каждой упорядоченной паре $(x,y)\in \mathbb{R}^2$ ставит в соответствие элемент из $\mathbb{R}$, называемый суммой $x+y$ и удовлетворяющий следующим аксиомам:
		\begin{enumerate}
			\item $\exists 0\in \mathbb{R}$, такой, что $\forall x\in \mathbb{R}:x+0=0+x=x$
			\item $\forall x \ \exists$ противоположный элемент ``$-x$'', такой, что $x+(-x)=(-x)+x=0$
			\item Ассоциативность
			\item Коммутативность
		\end{enumerate}
	\item На $\mathbb{R}$ определена операция умножения ``$\cdot$'', то есть $\forall(x,y)\in\mathbb{R}^2$ ставится в соответствие элемент $(x\cdot y)\in \mathbb{R}$.
		\begin{enumerate}
			\item $\exists $ нейтральный элемент $1\in \mathbb{R}$, такой, что $\forall x\in \mathbb{R}:1\cdot x=x\cdot 1=x$.
		\end{enumerate}
\end{enumerate}
	
\end{definition}

\subsection{Геометрическая интерпретация $\mathbb{R}$}
\subsection{Числовые промежутки}
\subsection{Бесконечные числовые промежутки}
\subsection{Окрестности точки}
\subsection{Принцип вложенных отрезков (Коши-Кантора)}
\subsection{Ограниченные и неограниченные числовые множества}
\subsection{Точные грани числового множества}
\subsection{Принцип Архимеда}


\section{Функции или отображения}
\subsection{Понятие функции}
\subsection{Ограниченные и неограниченные числовые множества}
\subsection{Обратные функции}
\subsection{Чётные и нечётные функции}
\subsection{Периодические функции}
\subsection{Сложная функция (композиция)}
\subsection{Основные элементарные функции}

\section{Числовые последовательности и их пределы}
\begin{definition}
	$f:\mathbb{N}\rightarrow \mathbb{R}$ - числовая последовательность, т.е. $\{x_n\}_{n=1}^\infty$, $x_n \in \mathbb{R}$.
\end{definition}

\subsection{Ограниченные и неограниченные числовые последовательности}

\begin{definition}
	Числовая последовательность $\{x_n\}_{n=1}^\infty$ называется 

1) ограниченной сверху, если $\exists M\in \mathbb{R}: \forall n\in \mathbb{N}: \ x_n\le M$;

2) ограниченной снизу, если \space $\exists M\in \mathbb{R}: \forall n\in \mathbb{N}: \ x_n\ge M$;

3) ограниченной, если \space\space\space\space\space\space\space\space\space $\exists M\in \mathbb{R}: \forall n\in \mathbb{N}:|x_n| \le M$;

4) неограниченной, если \space\space\space\space\space\space $\exists M\in \mathbb{R}: \forall n\in \mathbb{N}:|x_n|> M$;
\end{definition}

\subsection{Предел числовой последовательности}
\begin{definition}
	Число $a\in\mathbb{R}$ называется пределом числовой последовательности, если $\forall \varepsilon>0$ существует такой номер $n$, зависящий от $\varepsilon$, что $\forall$ натурального числа $N>n$ верно неравенство $|x_n - a| < \varepsilon$.
	$$\lim_{n\to\infty} x_n = a$$
\end{definition}

\subsection{Бесконечные пределы}
\subsection{Свойства сходящихся последовательностей}
\subsection{Монотонные числовые последовательности}
\subsection{Число $e$}
\subsection{Гиперболические функции}
\subsection{Предельные точки числового множества}
\subsection{Предельные точки числовых последовательностей}

\end{document} % конец документа
