\documentclass[a4paper,12pt]{article} % добавить leqno в [] для нумерации слева

%%% Работа с русским языком
\usepackage{cmap}					% поиск в PDF
\usepackage{mathtext} 				% русские буквы в фомулах
\usepackage[T2A]{fontenc}			% кодировка
\usepackage[utf8]{inputenc}			% кодировка исходного текста
\usepackage[english,russian]{babel}	% локализация и переносы

%%% Дополнительная работа с математикой
\usepackage{amsmath,amsfonts,amssymb,amsthm,mathtools} % AMS
\usepackage{icomma} % "Умная" запятая: $0,2$ --- число, $0, 2$ --- перечисление

%% Номера формул
\mathtoolsset{showonlyrefs=true} % Показывать номера только у тех формул, на которые есть \eqref{} в тексте.

%% Шрифты
\usepackage{euscript}	 % Шрифт Евклид
\usepackage{mathrsfs} % Красивый матшрифт
%% Акценты
\usepackage{accents}

%% Свои команды
\DeclareMathOperator{\sgn}{\mathop{sgn}}
% Теоремы
\newtheorem{definition}{Определение}[subsection]
\newtheorem{theorem}{Теорема}[subsection]
\newtheorem{corollary}{Следствие}[theorem]
\newtheorem{lemma}{Утверждение}[subsection]
\theoremstyle{remark}
\newtheorem*{remark}{Замечание}
%\newtheoremstyle{named}{}{}{\itshape}{}{\bfseries}{.}{.5em}{\thmnote{#3's }#1}
%\theoremstyle{named}
%\newtheorem*{namedtheorem}{Theorem}

%% Colors
\usepackage{xcolor}
\definecolor{nord8}{RGB}{136,192,208}
\definecolor{nord9}{RGB}{129,161,193}
\definecolor{nord10}{RGB}{94,129,172}
\definecolor{nord11}{RGB}{191,97,106}
\definecolor{nord12}{RGB}{208,135,112}
\definecolor{nord13}{RGB}{235,203,139}
\definecolor{nord14}{RGB}{163,190,140}
\definecolor{nord15}{RGB}{180,142,173}

\usepackage{textcomp}

%% TikZ
\usepackage{tikz}
\usetikzlibrary{arrows.meta}
\usetikzlibrary{shapes.geometric,positioning}
\usepackage{pgfplots}
%\usetikzlibrary{patterns.meta}

% figure support
\usepackage{import}
\usepackage{pdfpages}
\usepackage{transparent}

\newcommand{\incfig}[2][1]{%
    \def\svgwidth{#1\columnwidth}
    \import{./figures/}{#2.pdf_tex}
}

\pdfsuppresswarningpagegroup=1



%% Перенос знаков в формулах (по Львовскому)
\newcommand*{\hm}[1]{#1\nobreak\discretionary{}
{\hbox{$\mathsurround=0pt #1$}}{}}

%%% Заголовок
\author{Власова Елена Александровна}
\title{Лекции по математическому анализу для 1 курса ФН2, 3}
\date{2024-2025 год.}

\begin{document} % конец преамбулы, начало документа

\maketitle

\newpage
\tableofcontents
\newpage

\part*{Элементарные функции и их пределы}

\section{Введение}
\subsection{Элементы теории множеств}
\subsection{Кванторные операции}
\subsection{Метод математической индукции}



\newpage
\section{Множество действительных чисел}
\subsection{Аксиоматика действительных чисел}
\begin{definition}
	Множество $\mathbb{R}$ называется множеством действительных чисел, если элементы этого множества удовлетворяют следующему комплексу условий:
\begin{enumerate}
	\item На $\mathbb{R}$ определена операция сложения  ``$+$'', то есть задано отображение, которое каждой упорядоченной паре $(x, y)\in \mathbb{R}^2$ ставит в соответствие элемент из $\mathbb{R}$, называемый суммой $x+y$ и удовлетворяющий следующим аксиомам:
		\begin{enumerate}
			\item $\exists 0\in \mathbb{R}$, такой, что $\forall x\in \mathbb{R}:x+0=0+x=x$;
			\item $\forall x \ \exists$ противоположный элемент $-x$, такой, что $x+(-x)=(-x)+x=0$;
			\item Ассоциативность. $\forall x, y, z\in \mathbb{R}:(x+y)+z=x+(y+z)$;
			\item Коммутативность. $\forall x, y\in \mathbb{R}:x+y=y+x$.
		\end{enumerate}
	\item На $\mathbb{R}$ определена операция умножения ``$\cdot$'', то есть $\forall(x, y)\in\mathbb{R}^2$ ставится в соответствие элемент $(x\cdot y)\in \mathbb{R}$.
		\begin{enumerate}
			\item $\exists $ нейтральный элемент $1\in \mathbb{R}$, такой, что $\forall x\in \mathbb{R}:1\cdot x=x\cdot 1=x$;
			\item $\forall x\in \mathbb{R}\backslash\{0\} \ \exists $ обратный элемент ``$x^{-1}$'', такой, что $x\cdot x^{-1}=x^{-1}\cdot x=1$;
			\item Ассоциативность. $\forall x, y, z\in \mathbb{R}\backslash\{0\}:(x\cdot y)\cdot z=x\cdot (y\cdot z)$;
			\item Коммутативность. $\forall x, y\in \mathbb{R}\backslash\{0\}:x\cdot y=y\cdot x$.
		\end{enumerate}
\end{enumerate}
	Операция умножения дистрибутивна по отношению к сложению. 
		$$\forall x, y, z\in \mathbb{R}:(x+y)z=xz+yz$$
\begin{enumerate}
\setcounter{enumi}{2}
	\item Отношения порядка. Для $\mathbb{R}$ определено отношение ``$\le $ ''.
		\begin{enumerate}
			\item $\forall x\in \mathbb{R}:x\le x$;
			\item $\forall x, y\in \mathbb{R}:(x\le y \land y\le x)\implies x=y$;
				
				\ldots
		\end{enumerate}
\end{enumerate}
	
\end{definition}

\subsection{Геометрическая интерпретация $\mathbb{R}$}
\subsection{Числовые промежутки}
\subsection{Бесконечные числовые промежутки}



\subsection{Окрестности точки}
\begin{definition}
	Окрестностью точки $a \in \mathbb{R}$ называется любой интервал, содержащий точку $a$ и обозначается $U(a)$.
\end{definition}

Пусть $\varepsilon$ --- некоторое положительное число.
\begin{definition}
	$\varepsilon$-окрестностью точки $a\in \mathbb{R}$ называется интервал $(a-\varepsilon;a+\varepsilon)$ и обозначается $U_\varepsilon(a)$.
	\[
	c\in U_\varepsilon(a) \iff |a-c| < \varepsilon
	.\] 
\end{definition}

\begin{definition}
	Проколотой $\varepsilon$-окрестностью точки $a\in \mathbb{R}$ называется множество $(a-\varepsilon;a)\cup (a; a+\varepsilon) = U_\varepsilon(a)\ \backslash \{a\}$ и обозначается $\accentset{\circ}{U}_\varepsilon(a)$.
\end{definition}

\begin{definition}
	Окрестностью бесконечности называют любое множество вида $(-\infty; a)\cup (b; +\infty).$
\end{definition}
\begin{definition}
	$\varepsilon$-окрестностью бесконечности называют множество $(-\infty; -\varepsilon)\cup (\varepsilon; +\infty)$.

	Примечание: $U_\varepsilon(\infty)=\accentset{\circ}{U}_\varepsilon(\infty).$
\end{definition}

\subsection{Принцип вложенных отрезков (Коши-Кантора)}
\begin{definition}
	Пусть $\{x_n\}_{n=1}^{\infty}$ --- последовательность некоторых множеств. Если $\forall n\in \mathbb{N}:X_n\supset X_{n+1}$, то эта последовательность называется последовательностью вложенных отрезков.
\end{definition}

\subsection{Ограниченные и неограниченные числовые множества}
\subsection{Точные грани числового множества}
\subsection{Принцип Архимеда}


\newpage
\section{Функции или отображения}
\subsection{Понятие функции}
\subsection{Ограниченные и неограниченные числовые множества}
\subsection{Обратные функции}
\subsection{Чётные и нечётные функции}
\subsection{Периодические функции}
\subsection{Сложная функция (композиция)}
\subsection{Основные элементарные функции}



\newpage
\section{Числовые последовательности и их пределы}



\begin{definition}
	$f:\mathbb{N}\rightarrow \mathbb{R}$ --- числовая последовательность, т.е. $\{x_n\}_{n=1}^\infty$, $x_n \in \mathbb{R}$.
\end{definition}



\subsection{Ограниченные и неограниченные числовые последовательности}

\begin{definition}
	Числовая последовательность $\{x_n\}_{n=1}^\infty$ называется 

	\begin{enumerate}
		\item ограниченной сверху, если $\exists M\in \mathbb{R}: \forall n\in \mathbb{N}: \ x_n\le M$;
		\item ограниченной снизу, если \space $\exists M\in \mathbb{R}: \forall n\in \mathbb{N}: \ x_n\ge M$;
		\item ограниченной, если \space\space\space\space\space\space\space\space\space $\exists M\in \mathbb{R}: \forall n\in \mathbb{N}:|x_n| \le M$;
		\item неограниченной, если \space\space\space\space\space\space $\exists M\in \mathbb{R}: \forall n\in \mathbb{N}:|x_n|> M$;
	\end{enumerate}
\end{definition}



\subsection{Предел числовой последовательности}
\begin{definition}
	Число $a\in\mathbb{R}$ называется пределом числовой последовательности, если $\forall \varepsilon>0$ существует такой номер $n$, зависящий от $\varepsilon$, что $\forall$ натурального числа $N>n$ верно неравенство $|x_n - a| < \varepsilon$.
	$$\lim_{n\to\infty} x_n = a$$
\end{definition}
\noindent Пример:
\[
\lim_{n \to \infty} \frac{1}{n} = 0 \iff \forall \varepsilon>0 \exists N=N(\varepsilon) \in \mathbb{N} \quad \forall n>N : \frac{1}{n} < \varepsilon
.\] 
$\frac{1}{n} < \varepsilon \implies n > \frac{1}{\varepsilon}$. Возьмем $N(\varepsilon) = [\frac{1}{\varepsilon}]$. Тогда $\forall n>[\frac{1}{\varepsilon}] : \frac{1}{n} < \varepsilon.$


\begin{definition}
	Если $\{x_n\}_{n=1}^{\infty}$ имеет конечный предел $a$, то эта последовательность называется сходящейся, в противном случае --- расходящейся.
\end{definition}

\begin{definition}
	Если $\lim_{n \to \infty} x_n = 0$, то последовательность $\{x_n\}_{n=1}^{\infty}$ называется бесконечно малой (б.м.).
\end{definition}
\subsection{Бесконечные пределы}



\subsection{Свойства сходящихся последовательностей}
\begin{theorem}[о единственности предела]
	Любая сходящаяся последовательность имеет только один предел.	
\end{theorem}
\begin{proof}
	``От противного''. Пусть $\{x_n\}_{n=1}^{\infty}$ --- сходящаяся последовательность. Предположим, что $\exists \lim_{n \to \infty} x_{n} = a$ и $\exists \lim_{n \to \infty} x_{n} = b$, причем $a\neq b$. Пусть для определенности $a<b$.
	\[
	\lim_{n \to \infty} x_{n} = a \iff \forall \varepsilon>0 \quad \exists N_1=N_1(\varepsilon)\in \mathbb{N} : \forall n>N_1 : |x_{n}-a|<\varepsilon
	,\] 
	\[
	\lim_{n \to \infty} x_{n} = b \iff \forall \varepsilon>0 \quad \exists N_2=N_2(\varepsilon)\in \mathbb{N} : \forall n>N_2 : |x_{n}-b|<\varepsilon
	.\] 
	\[
		N = \max\{N_1, n_2\} \implies \forall n>N : 
		\begin{cases}
			|x_{n}-a| < \varepsilon, \\
			|x_{n}-b| < \varepsilon.
		\end{cases}
	\] 
\begin{center}
	\begin{tikzpicture}
		% Числовая прямая x.
		\draw[-{Classical TikZ Rightarrow[length=1mm]}] (0,0) -- (12,0);	
		\draw (12,0) node[anchor=north] (x) {$x$};
		% Точки
		\draw[shift={(3,0)},color=black] (0pt,3pt) -- (0pt,-3pt) node[anchor=north] (a) {$a$};
		\draw[shift={(9,0)},color=black] (0pt,3pt) -- (0pt,-3pt) node[anchor=north] (b) {$b$};
		\draw (1,0) node[anchor=north] (a-e) {$a-\varepsilon$};
		\draw (5,0) node[anchor=north] (a+e) {$a+\varepsilon$};
		\draw (7,0) node[anchor=north] (b-e) {$b-\varepsilon$};
		\draw (11,0) node[anchor=north] (b+e) {$b+\varepsilon$};
		\draw (6,2) node[anchor=north, yshift=0cm] (xn) {$\forall n>N:x_{n}$};
		% Интервалы
		\draw (1,0) to[out=70,in=110] (5,0);
		\draw (7,0) to[out=70,in=110] (11,0);
		% Стрелки
		\draw[->] (6.8,1.4) to (3.6,0.5);
		\draw[->] (6.8,1.4) to (8.4,0.5);
	\end{tikzpicture}
\end{center}
Выберем $\varepsilon = \frac{b-a}{4} > 0$. Найдем $N_1(\varepsilon), N_2(\varepsilon), N=\max\{N_1, N_2\}$, тогда
\[
	\forall n>N \quad |x_{n}-a| < \frac{b-a}{4}, \quad |x_{n}-b| < \frac{b-a}{4}
.\] 
Следовательно,
\[
	0 < b-a = |b-a| = |b-x_{n}+x_{n}-a| \le |x_{n}-b|+|x_{n}-a| < \frac{b-a}{2}
,\] 
то есть
\[
0 < b-a < \frac{b-a}{2}
.\] 
Мы пришли к противоречию, следовательно, $a=b \implies \{x_n\}_{n=1}^{\infty}$ имеет единственный предел.
\end{proof}

\begin{theorem}[об ограниченности сходящейся последовательности]
	Любая сходящаяся последовательность является ограниченной.	
	\begin{center}
		\begin{tikzpicture}
			% Числовая прямая 
				\draw[-{Classical TikZ Rightarrow[length=1mm]}] (0,0) -- (12,0);        
			% Точки
				\filldraw[black] (4,0) circle (0.05) node[anchor=north] (x_1) {$x_1$}; 
				\filldraw[black] (10.3,0) circle (0.05) node[anchor=north] (x_2) {$x_2$}; 
				\filldraw[black] (5.3,0) circle (0.05) node[anchor=north] (x_3) {$x_3$}; 
				\filldraw[black] (7.8,0) circle (0.05) node[anchor=north] (x_n) {$x_n$}; 
				\draw[shift={(6,0)},color=black] (0pt,3pt) -- (0pt,-3pt) node[anchor=north] (a) {$a$};
				\draw (3,0) node[anchor=north] (a-e) {$a-\varepsilon$};
				\draw (9,0) node[anchor=north] (a+e) {$a+\varepsilon$};
			% Окрестность
				\draw (8,2) node[anchor=north] (Ua) {$U(a)$};
			% Интервалы 
				\draw (3,0) to[out=70,in=110] (9,0);
			% Отрезок
				\draw (1,0) -- ++(0,2) -- ++(10,0) -- ++(0,-2);
		\end{tikzpicture}
	\end{center}
\end{theorem}
\begin{proof}
	Если $\{x_n\}_{n=1}^{\infty}$ сходится, то 
	\[
	\exists \lim_{n \to \infty} = a\in \mathbb{R} \implies \forall \varepsilon>0 \quad \exists N=N(\varepsilon)\in \mathbb{N} \quad \forall n>N : |x_{n}-a|<\varepsilon
	\]
	Пусть $\varepsilon = 1 \implies \exists N=N(1) \quad \forall n>N : |x_{n}-a| < 1$. Следовательно,
	\[
	|x_{n}| = |x_{n}-a+a| \le |x_{n}-a|+|a| < 1+|a|
	.\] 
	Пусть $M_0=1+|a| \implies \forall n>N : x_{n} < M_0$. 
	
	\noindent Пусть $M=\max\{|x_1|, |x_2|, \ldots, |x_{n}|, M_0\}$, тогда $\forall n\in \mathbb{N} : x_{n}\le M\implies \{x_n\}_{n=1}^{\infty}$ является ограниченной.
\end{proof}

\begin{remark}
	Ограниченность является необходимым условием сходимости числовой последовательности. В то же время условие ограниченности не является достаточным для сходимости числовой последовательности.
	Например, $\{(-1)^n\}_{n=1}^{\infty}$ --- ограниченная, но не сходящаяся числовая последовательность.
\end{remark}



\subsection{Монотонные числовые последовательности}
\begin{definition}
	Числовая последовательность $\{x_n\}_{n=1}^{\infty}$ называется 
	\begin{enumerate}
		\item возрастающей, если \ \ $\forall n\in \mathbb{N}:x_{n}<x_{n+1}$;
		\item убывающей, если \ \ \ \ \ \ \ $\forall n\in \mathbb{N}:x_{n}>x_{n+1}$;
		\item неубывающей, если \ \ \ \ $\forall n\in \mathbb{N}:x_{n}\le  x_{n+1}$;
		\item невозрастающей, если $\forall n\in \mathbb{N}:x_{n}\ge x_{n+1}$
	\end{enumerate}
\end{definition}
Для монотонных числовых последовательностей ограниченность является достаточным условием для сходимости.
\begin{theorem}[Вейерштрасса о сходимости монотонных числовых последовательностей]
	Если последовательность не убывает и ограничена сверху, то она является сходящейся.
	Если последовательность не возрастает и ограничена снизу, то она является сходящейся. В общем, любая монотонная последовательность сходится.
\end{theorem}
\begin{proof}
	Пусть $\{x_n\}_{n=1}^{\infty}$ не убывает и ограничена сверху $\implies$ \\
	$\implies \exists M\in \mathbb{R} :\forall n\in \mathbb{N}:x_{n}\le M\implies$ \\
	\indent $\implies$ множество значений этой последовательности \\
	\indent \indent \ \ \ $\{x_1, x_2,\ldots, x_{n}, \ldots\}=A$ является ограниченным \\ 
	\indent \indent \ \ \ сверху числовым множеством $\implies$ \\
	\indent \indent \indent \indent \indent \indent $\implies \exists \sup A\in \{x_n\}_{n=1}^{\infty}=a$, то есть
	\begin{enumerate}
		\item $\forall n\in \mathbb{N}:x_{n}\le a$;
		\item $\forall \varepsilon>0 \ \exists N=N(\varepsilon)\in \mathbb{N}:x_{N}>a-\varepsilon$.
	\end{enumerate}

\begin{center}
	\begin{tikzpicture}
		\draw[-{Classical TikZ Rightarrow[length=1mm]}] (0,0) -- (12,0);	
		% Числовая прямая
		\filldraw[black] (4.5,0) circle (0.05) node[anchor=south]{$x_N$}; %node[yshift=-1.4mm]
		% Точки
		%\draw[shift={(5,0)},color=black] (0pt,3pt) -- (0pt,-3pt) node[anchor=north] (a) {$a$};
		\draw (6,0) node[anchor=north] (a) {$a$};
		\draw (2,0) node[anchor=north] (a-ep) {$a-\varepsilon$};
		\draw (10,0) node[anchor=north] (a+ep){$a+\varepsilon$};
		%\draw[shift={(2,0)},color=black] (0pt,3pt) -- (0pt,-3pt) node[anchor=north] (a-ep) {$a-\varepsilon$};
		%\draw[shift={(8,0)},color=black] (0pt,3pt) -- (0pt,-3pt) node[anchor=north] (a+ep){$a+\varepsilon$};
		% Интервалы
		\draw (2,0) to[out=70,in=110] (6,0);
		\draw (6,0) to[out=70,in=110] (10,0);
	\end{tikzpicture}
\end{center}
$\{x_n\}_{n=1}^{\infty}$ --- неубывающая последовательность, то есть 
\begin{multline}
\forall n>N=N(\varepsilon):x_{n}\ge x_N \implies \\
\implies a-\varepsilon<x_N\le x_{n}\le a<a+\varepsilon\implies \\
\implies a-\varepsilon<x_{n}<a+\varepsilon \implies |x_{n}-a|<\varepsilon \implies \\
\implies \forall \varepsilon>0 \quad \exists N=N(\varepsilon)\in \mathbb{N}:\forall n>N:|x_{n}-a|<\varepsilon \implies \\
\implies \exists \lim_{n \to \infty}x_{n} = a\in \mathbb{R} \implies \{x_n\}_{n=1}^{\infty} \text{ сходится.}
\end{multline}
Если $\{x_n\}_{n=1}^{\infty}$ --- невозрастающая и ограниченная снизу последовательность, то 
\[
\exists \lim_{n \to \infty} x_{n}=\inf A, A = \{x_1, x_2, \ldots, x_{n}, \ldots\} 
.\] 
Доказательство аналогично.
\end{proof}

\subsection{Число $e$}

\subsection{Гиперболические функции}
\subsection{Предельные точки числового множества}
\begin{definition}
	Точка $a\in \mathbb{R}$ называется предельной точкой множества $X\subset \mathbb{R} \iff$ любая окрестность $U(a)$ содержит бесконечно много элементов множества $X$.
\end{definition}
\begin{remark}
	Множество $A$ называется бесконечным или содержащим бесконечно много элементов, если при вычитании из $A$ любого его конечного подмножества получается непустое множество.
\end{remark}

Множество всех предельных точек множества $X$ называется производным множеством для $X$ и обозначается $X'$.

\begin{lemma}
	Точка $a\in \mathbb{R}$ является предельной для $X\subset \mathbb{R}\iff$ в любой проколотой $\delta$-окрестности точки $a$ содержится хотя бы один элемент множества $X$, т.е.
	\[
		\forall \delta>0 \quad \exists x\in X \cap \accentset{\circ}{U}(a)
	.\]
\end{lemma}
\begin{proof}
	$(\implies)$ Необходимость.
	
	$a$ --- предельная для $X\subset \mathbb{R}\implies$ \\ 
	\indent $\implies$ любая $U(a)$ содержит бесконечно много элементов из $X \implies$ \\ 
	\indent \indent $\implies \accentset{\circ}{U}(a)$ тоже содержит бесконечно много элементов из $X\implies$ \\ 
	\indent \indent \indent $\implies$ любая $\accentset{\circ}{U}$ содержит хотя бы один элемент $x\in X$.

	\noindent $(\impliedby)$ Достаточность.
	\[
		\forall \delta>0 \quad \exists x\in X \cap \accentset{\circ}{U}(a)
	.\]
	Выберем любую $U(a)$. Тогда
\[
\exists \delta_1>0 : \accentset{\circ}{U}(a)\subset U(a) \implies \exists x_1\in X : x_1\in \accentset{\circ}{U}_{\delta_1}(a)
.\] 
\begin{center}
	\begin{tikzpicture}
		\draw[-{Classical TikZ Rightarrow[length=1mm]}] (0,0) -- (12,0);	
		% Числовая прямая
		\filldraw[black] (3.7,0) circle (0.05) node[anchor=south]{$x_1$};
		% Точки
		\draw (6,0) node[anchor=north] (a)   {$a$};
		\draw (1,0) node[anchor=north] (b)   {$b$};
		\draw (11,0) node[anchor=north] (c)   {$c$};
		\draw (3,0) node[anchor=north] (a-d) {$a-\delta_1$};
		\draw (9,0) node[anchor=north] (a+d) {$a+\delta_1$};
		\draw (7,0) node[anchor=north, yshift=3.4cm] (Ua) {$U(a)$};
		\draw (5,0) node[anchor=north, yshift=1.6cm] (Uo) {$\accentset{\circ}{U}_{\delta_1}(a)$};
		% Интервалы
		\draw (1,0) to[out=70,in=110] (11,0);
		\draw (3,0) to[out=70,in=110] (6,0);
		\draw (6,0) to[out=70,in=110] (9,0);
	\end{tikzpicture}
\end{center}
Пусть $\delta_2 = \frac{|x_1-a|}{2} > 0$. Тогда
\[
	\exists x_2 \in \accentset{\circ}{U}_{\delta_2}(a) : x_2\neq x_1
.\] 

\begin{center}
	\begin{tikzpicture}
		% Числовая прямая
		\draw[-{Classical TikZ Rightarrow[length=1mm]}] (0,0) -- (12,0);	
		% Точки
		\filldraw[black] (2.1666,0) circle (0.05) node[anchor=south]{$x_1$}; 
		\filldraw[black] (5,0) circle (0.05) node[anchor=south]{$x_2$}; 
		\draw (6,0) node[anchor=north] (a)   {$a$};
		\draw (1,0) node[anchor=north] (a-d) {$a-\delta_1$};
		\draw (11,0) node[anchor=north] (a+d) {$a+\delta_1$};
		\draw (3.8334,0) node[anchor=north] (a-d) {$a-\delta_2$};
		\draw (8.1666,0) node[anchor=north] (a+d) {$a+\delta_2$};
		\draw (9.5,0) node[anchor=north, yshift=2.1cm] (Uo_1) {$\accentset{\circ}{U}_{\delta_1}(a)$};
		\draw (8.1,0) node[anchor=north, yshift=1.25cm] (Uo_2) {$\accentset{\circ}{U}_{\delta_2}(a)$};
		% Интервалы
		\draw (1,0) to[out=80,in=100] (6,0);
		\draw (6,0) to[out=80,in=100] (11,0);
		\draw (3.8334,0) to[out=70,in=110] (6,0);
		\draw (6,0) to[out=70,in=110] (8.1666,0);
	\end{tikzpicture}
\end{center}
Пусть $\delta_3 = \frac{|x_2-a|}{2} > 0$. Тогда 
\[
	\exists x_3 \in \accentset{\circ}{U}_{\delta_3}(a) : x_3\neq x_2 
\] 
и т.д. На шаге $n$:
\[
\delta_n = \frac{|x_{n-1}-a|}{2} > 0 \implies \exists x_n \in \accentset{\circ}{U}_{\delta_n}(a) : x_n\neq x_{k}, k =1, 2, \ldots, n-1 
.\] 
Таким образом,
\[
	\exists \{x_n\}_{n=1}^{\infty}\in U(a) : x_{n}\in X, x_{n}\neq x_k, n\neq k,
\] 
а значит, любая окрестность $U(a)$ содержит бесконечно много элементов из $X \implies a$ --- предельная точка.
\end{proof}

\begin{lemma}
	Если точка $a\in \mathbb{R}$ является предельной точкой для множества $X\subset \mathbb{R}$, то
	\[
	\exists \{x_n\}_{n=1}^{\infty}\subset X : \lim_{n \to \infty} x_n = a
	.\] 
\end{lemma}

\begin{proof}
	$a$ --- предельная точка для $X\subset \mathbb{R} \iff \forall \delta>0 \quad \accentset{\circ}{U}_\delta(a)$ содержит хотя бы одну точку множества $X$ (по утверждению 1).
	$\newline$
	Выберем $\{\delta_n\}_{n=1}^{\infty}, \delta_n=\frac{1}{n}>0$, тогда
	\[
		\forall n\in \mathbb{N} \quad \exists x_n\in X : x_{n}\in \accentset{\circ}{U}_{\delta_n}(a)
	,\] 
	то есть 
	\[
	0 < |x_n-a| < \frac{1}{n}
	.\] 
Т.к. $\lim_{n \to \infty} \frac{1}{n} = 0$, 
\[
	\forall \varepsilon>0 \quad \exists N=N(\varepsilon)\in \mathbb{N} \quad \forall n>N : \frac{1}{n} < \varepsilon
,\] 
а значит,
\[
|x_{n}-a| < \frac{1}{n} < \varepsilon \implies \lim_{n \to \infty} x_{n} = a
.\] 
\end{proof}

\begin{theorem}[принцип Больцано-Вейерштрасса]
	Любое ограниченное бесконечное числовое множество имеет хотя бы одну предельную точку.
\end{theorem}
\begin{proof}
	Пусть $X$ --- бесконечное ограниченное множество, то есть $\exists I_1 =[a_1, b_1] : X\subset [a_1, b_1]$.
	Пусть $c_1=\frac{a_1+b_1}{2}$, т.е. середина отрезка $I_1$.
	\begin{center}
		\begin{tikzpicture}
			%\fill[nord8] (1,0)rectangle(11,1); 
			%% Числовая прямая x.
			\draw[-{Classical TikZ Rightarrow[length=1mm]}] (0,0) -- (12,0);	
			\draw (12,0) node[anchor=north] (x) {$x$};
			%% a1, b1, c1
			\filldraw[black] (1,0) circle (0.05) node[anchor=north] (a1) {$a_1$};
			\filldraw[black] (11,0) circle (0.05) node[anchor=north] (b1) {$b_1$};
			\draw[shift={(6,0)},color=black] (0pt,3pt) -- (0pt,-3pt) node[anchor=north] (c1) {$c_1$};
			\draw (6,0) node[anchor=north, yshift=1.7cm] (I_1) {$I_1$};
			%% X and arrow
			\draw (4,-1) node[anchor=north] (X) {$X$};
			\draw[->] (X) to[out=135,in=225] (4,0.5);
			%% Отрезок
			\draw (1,0) -- ++(0,1) -- ++(10,0) -- ++(0,-1);
		\end{tikzpicture}
	\end{center}
	Так как множество $X$ бесконечное, то либо отрезок $[a_1, c_1]$, либо отрезок $[c_1, b_1]$ содержит бесконечно много элементов множества $X$. Обозначим ту половину отрезка $I_1$, которая содержит бесконечно много элементов множества $X$ через $I_2 = [a_2, b_2], I_2\subset I_1$. Выразим длину отрезка $I_2$:
\[
	|I_2| = b_2-a_2 = \frac{b_1-a_1}{2} = \frac{|I_1|}{2}
.\] 
На отрезке $I_2$ содержится бесконечно много элементов множества $X$.
Пусть $c_2=\frac{a_2+b_2}{2}$ --- середина $I_2$, тогда либо $[a_2, c_2]$, либо $[c_2, b_2]$ содержит бесконечно много элементов множества $X$. Обозначим ту половину $I_2$, где бесконечно много элементов множества $X$ через $I_3 = [a_3, b_3]$. Тогда
\[
|I_3| = \frac{|I_1|}{2^2}
\] 
и т.д. На шаге n: $I_n=[a_n, b_n], c_n = \frac{a_n + b_n}{2}$ --- середина $I_n$, $I_n$ содержит бесконечно много элементов из $X$, тогда либо $[a_n, c_n]$, либо $[c_n, b_n]$ содержит бесконечно много элементов из $X \implies I_{n+1}=[a_{n+1}, b_{n+1}]\subset I_n$ и содержит бесконечно много элементов из $X$. Таким образом, мы получили последовательность вложенных отрезков $\{I_n\}_{n=1}^{\infty} : I_1\supset I_2\supset \ldots\supset I_n\supset I_{n+1}\supset \ldots$
\begin{multline}
	|I_n|=\frac{|I_1|}{2^{n-1}} \implies \lim_{n \to \infty} \frac{|I_1|}{2^{n-1}} = 0 \implies \\
	\implies \forall \varepsilon>0 \quad \exists N=N(\varepsilon)\in \mathbb{N} \quad \forall n>N : |I_n| < \varepsilon.
\end{multline}
По принципу Коши-Кантора $\exists !$ общая точка $c$, т.е. $\forall n\in \mathbb{N} : c\in I_n$.
\[
	\forall U(c) \quad \exists \varepsilon>0 \quad U_\varepsilon(c) \subset U(c) \implies \exists n\in \mathbb{N} : I_n=[a_n, b_n] \subset U_\varepsilon(c)
\] 
(например, $|I_n| < \frac{\varepsilon}{2}$).

\begin{center}
	\begin{tikzpicture}
		% Числовая прямая
		\draw[-{Classical TikZ Rightarrow[length=1mm]}] (0,0) -- (12,0);	
		% Точки
		\draw[shift={(6,0)},  color=black] (0pt,3pt) -- (0pt,-3pt) node[anchor=north] (c)   {$c$};
		\filldraw[black] (4.5,0) circle (0.05) node[anchor=north] {$a_n$}; 
		\filldraw[black] (7.5,0) circle (0.05) node[anchor=north] {$b_n$}; 
		%%\draw[shift={(4.5,0)},color=black] (0pt,3pt) -- (0pt,-3pt) node[anchor=north] (a_n) {$a$};
		%%\draw[shift={(7.5,0)},color=black] (0pt,3pt) -- (0pt,-3pt) node[anchor=north] (b_n) {$b$};
		\draw (3,0) node[anchor=north] (a-d) {$c-\varepsilon$};
		\draw (9,0) node[anchor=north] (a+d) {$c+\varepsilon$};
		\draw (7,0) node[anchor=north, yshift=3.4cm] (Uc) {$U(c)$};
		% Интервалы
		\draw (1,0) to[out=70,in=110] (11,0);
		\draw (3,0) to[out=70,in=110] (9,0);
		\draw (4.5,0) -- ++(0,0.7) -- ++(3,0) -- ++(0,-0.7);
	\end{tikzpicture}
\end{center}
Отрезок $I_n$ содержит бесконечно много элементов множества $X$ по построению последовательности $\{I_n\}_{n=1}^{\infty} \implies$ окрестность $U(c)$ содержит бесконечно много элементов из $X \implies c$ --- предельная.
\end{proof}



\subsection{Предельные точки числовых последовательностей}
\begin{definition}
	Точка $a\in \mathbb{R}$ называется предельной точкой числовой последовательно $\{x_n\}_{n=1}^{\infty} \iff$ любая окрестность $U(a)$ содержит бесконечно много элементов последовательности $\{x_n\}_{n=1}^{\infty} $.
\end{definition}
\begin{remark}
	Если $a$ --- предельная точка $\{x_n\}_{n=1}^{\infty} $, то любая $U(a)$ содержит какую-либо подпоследовательность $\{x_n\}_{n=1}^{\infty} $.
\end{remark}
Пример: $\{x_n\}_{n=1}^{\infty}, x_{n}=(-1)^{n}$. 
\begin{center}
	\begin{tikzpicture}
		\draw[-{Classical TikZ Rightarrow[length=1mm]}] (0,0) -- (12,0);		
		\filldraw[black] (4,0) circle (0.05) node[anchor=south] (x_{2n-1}) {$x_{2n-1}$};
		\filldraw[black] (8,0) circle (0.05) node[anchor=south] (x_{2n}) {$x_{2n}$};
		\draw 		 (4,0) node[anchor=north] (-1) {$-1$};
		\draw 		 (8,0) node[anchor=north] (1) {$1$};
	\end{tikzpicture}
\end{center}

\begin{theorem}
	Точка $a\in \mathbb{R}$ является предельной для $\{x_n\}_{n=1}^{\infty} \iff \exists \{x_{n_k}\}_{k=1}^{\infty} : \lim_{k \to \infty} x_{n_k} = a$.
\end{theorem}
\begin{proof}
	Докажем необходимость. Пусть $a$ --- предельная точка последовательности $\{x_n\}_{n=1}^{\infty}$. Выберем $\{\varepsilon_n\}_{n=1}^{\infty}, \varepsilon_n = \frac{1}{n}>0$.

	Для $n=1 \ \ U_{\varepsilon_{1}=1}(a)$ содержит $\infty$ много элементов $\implies \exists x_{n_1}\in U_{\varepsilon_1}(a)$, т.е. $|x_{n_1}-a| < 1$.

	Для $n=2 \ \ U_{\varepsilon_{2}=\frac{1}{2}}(a)$ содержит $\infty$ много элементов $\implies \exists n_2>n_1 : x_{n_2}\in U_{\varepsilon_2}(a)$, т.е. $|x_{n_2}-a| < \frac{1}{2}$. 

	Для $n=3 \ \ U_{\varepsilon_{3}=\frac{1}{3}}(a)$ содержит $\infty$ много элементов $\implies \exists n_3>n_2 : x_{n_3}\in U_{\varepsilon_3}(a)$, т.е. $|x_{n_3}-a| < \frac{1}{3}$ и т.д. 

	Для $n=k \ \ U_{\varepsilon_{k}=\frac{1}{k}}(a)$ содержит $\infty$ много элементов $\implies \exists n_k>n_{k-1} : x_{n_k}\in U_{\varepsilon_k}(a)$, т.е. $|x_{n_k}-a| < \frac{1}{k} \implies \{x_{n_k}\}_{k=1}^{\infty}$ является подпоследовательностью последовательности $\{x_n\}_{n=1}^{\infty} \implies \forall k\in \mathbb{N} : |x_{n_k} - a| < \frac{1}{k}$.
	\begin{multline}
	\lim_{k \to \infty} \frac{1}{k} = 0 \implies \forall \varepsilon>0 \quad \exists N=N(\varepsilon)\in \mathbb{N} \quad \forall k>N : \frac{1}{k} < \varepsilon \implies \\
	\implies \forall k > N \quad |x_{n_k}-a| < \frac{1}{k} < \varepsilon \implies \exists \lim_{k \to \infty} x_{n_k} = a.
	\end{multline}

	Докажем достаточность.

	Пусть $\exists \{x_{n_k}\}_{k=1}^{\infty} : \lim_{k \to \infty} x_{n_k} = a$. Выберем любую $U(a)$ и найдем такое $\varepsilon>0$, что $U_\varepsilon(a) \subset  U(a)$:
	\[
	\exists N=N_(\varepsilon)\in\mathbb{N} \quad \forall k>N : |x_{n_k}-a|<\varepsilon \implies x_{n_k}\in U_\varepsilon(a)\subset U(a)
	.\]
Следовательно, $U(a)$ содержит бесконечно много элементов $\{x_n\}_{n=1}^{\infty}$, а значит, $a$ --- предельная.
\end{proof}

\begin{theorem}
	Если $\exists \lim_{n \to \infty} x_n = a$, то $a$ является предельной точкой для $\{x_n\}_{n=1}^{\infty}$, причем единственной.
\end{theorem}

\begin{proof}
	$a$ --- предельная, если $\lim_{n \to \infty} x_n = a$ (по теореме 1).

Докажем единственность предельной точки для $\{x_n\}_{n=1}^{\infty}$ ``от противного''. Пусть $\exists b\neq a$, $b$ --- предельная точка $\{x_n\}_{n=1}^{\infty}$, тогда $|b-a|\ge \delta>0$. Т.к. $a=\lim_{n \to \infty} x_n$, любая $\varepsilon$-окрестность точки содержит бесконечно много элементов $\{x_n\}_{n=1}^{\infty}$, а именно все, начиная с номера $N(\varepsilon) + 1$, т.е. $\forall n>n(\varepsilon)$. Вне $U_\varepsilon(a)$ может содержаться не более конечного числа элементов $\{x_n\}_{n=1}^{\infty}$ (возможно $x_n$ с номерами $1, 2, \ldots, N(\varepsilon)$).

Выберем $\varepsilon=\frac{\delta}{4} > 0$. Тогда $\forall n>N(\varepsilon) : x_n \in  U_\varepsilon(a)$. 

\begin{center}
	\begin{tikzpicture}
		\draw[-{Classical TikZ Rightarrow[length=1mm]}] (0,0) -- (12,0);
		\draw[shift={(3.5,0)},color=black] (0pt,3pt) -- (0pt,-3pt) node[anchor=north] (b) {$b$};
		\draw[shift={(8.5,0)},color=black] (0pt,3pt) -- (0pt,-3pt) node[anchor=north] (a) {$a$};
		\draw (2,0) to[out=70,in=110] (5,0);
		\draw (7,0) to[out=70,in=110] (10,0);
		\draw            (4,1.5) node[anchor=north] (U_b) {$U_\varepsilon(b)$};
		\draw            (9,1.5) node[anchor=north] (U_a) {$U_\varepsilon(a)$};
		\draw (7,-1) node[anchor=north] (x_n) {$x_n, n>N(\varepsilon)$};
		\draw[->] (x_n) to (8,0.4);
	\end{tikzpicture}
\end{center}

Но $U_\varepsilon(a)\cap U_\varepsilon(b) = \varnothing$, что противоречит тому, что $b$ --- предельная точка для $\{x_n\}_{n=1}^{\infty}$, т.е. $U_\varepsilon(b)$ должна содержать бесконечно много элементов $\{x_n\}_{n=1}^{\infty}$, а туда может попасть не более конечного. Следовательно, $a = b$.
\end{proof}

\begin{theorem}
	Любая ограниченная числовая последовательность имеет хотя бы одну предельную точку.
\end{theorem}
\begin{proof}
	$\{x_n\}_{n=1}^{\infty}$ --- ограниченная, $X = \{x_1, x_2, \ldots, x_n, \ldots\} \subset \mathbb{R}$, $X$ --- множество значений числовой последовательность $\{x_n\}_{n=1}^{\infty}$. Т.к. $\{x_n\}_{n=1}^{\infty}$ --- ограниченная числовая последовательность, $X$ --- ограниченное числовое множество.
	Рассмотрим два случая.

	Первый: $X$ --- бесконечное числовое множество. Тогда $X$ по принципу Больцано-Вейерштрасса имеет хотя бы одну предельную точку $a$, т.е. в любую $U(a)$ попадает бесконечно много элементов множества $X$, а значит, и бесконечно много элементов $\{x_n\}_{n=1}^{\infty}$. Следовательно, $a$ --- предельная точка последовательности $\{x_n\}_{n=1}^{\infty}$.

	Второй: $X$ --- конечное числовое множество. Тогда хотя бы один элемент последовательности $\{x_n\}_{n=1}^{\infty}$ повторяется бесконечно много раз, т.е. $\exists$ подпоследовательность $\{x_{n_k}\}_{k=1}^{\infty}$ (постоянная $\forall k\in \mathbb{N}$), $x_{n_k} = a\in  X \implies$ a --- предельная точка $\{x_n\}_{n=1}^{\infty}, \lim_{k \to \infty} x_{n_k} = a$.
\end{proof}

\begin{theorem}
	Для того, чтобы точка $a\in \mathbb{R}$ была пределом $\{x_n\}_{n=1}^{\infty}$, т.е. $\lim_{n \to \infty} x_n = a$, необходимо и достаточно, чтобы $\{x_n\}_{n=1}^{\infty}$ была ограниченной и имела единственную предельную точку.
\end{theorem}
\begin{proof}
	Докажем необходимость.
	$\exists \lim_{n \to \infty} x_n = a \in \mathbb{R} \implies$ $\{x_n\}_{n=1}^{\infty}$ ограниченна (по свойству сходящейся последовательности), а значит, $\{x_n\}_{n=1}^{\infty}$ имеет единственную предельную точку (по теореме 2).

	Докажем достаточность. Пусть $a$ --- единственная предельная точка ограниченной последовательности $\{x_n\}_{n=1}^{\infty}.$. Докажем, что $\lim_{n \to \infty} x_n = a$.

	Т.к. $a$ --- .
\end{proof}



\subsection{Фундаментальные последовательности}
\begin{definition}
	Последовательность $\{x_n\}_{n=1}^{\infty} \subset  \mathbb{R}$ называется фундаментальной $\iff $
	\[
	\forall \varepsilon>0 \quad \exists N=N(\varepsilon) \in  \mathbb{N} \quad \forall n>N, \forall m>N : |x_n-x_m|<\varepsilon
	.\] 
\end{definition}

\begin{theorem}
	Если числовая последовательность $\{x_n\}_{n=1}^{\infty}$ фундаментальна, то она ограниченна.
\end{theorem}
\begin{proof}
	$\{x_n\}_{n=1}^{\infty}$ фундаментальная $\implies$
	\[
	\forall \varepsilon>0 \quad \exists N=N(\varepsilon)\in \mathbb{N} \quad \forall n>N, \forall m>N : |x_n-x_m| < \varepsilon
	.\] 
	Пусть $\varepsilon=1$, тогда
	\begin{multline}
	\exists N=N(1) \implies \forall n>N, m=N+1 : |x_n - x_{N+1}|<1 \implies \\
	\implies \forall n>N : |x_n| = |x_n - x_{N+1} + x_{N+1}| \le \\
	\le |x_n - x_{N+1}| + |x_{N+1}| < 1 + |x_{N+1}| = M_0 \implies \\
	\implies \forall n>N : |x_n| < M_0.
	\end{multline}
	Пусть $M = \max \{|x_1|, |x_2|, \ldots, |x_N|, M_0\}$, тогда $\forall n\in \mathbb{N} : |x_n| \le M \implies \{x_n\}_{n=1}^{\infty}$ ограниченна.
\end{proof}
\begin{center}
	\begin{tikzpicture}
		\draw[-{Classical TikZ Rightarrow[length=1mm]}] (0,0) -- (12,0);		
		\filldraw[black] (4,0) circle (0.05) node[anchor=south] (x_{2n-1}) {$x_{2n-1}$};
		\filldraw[black] (8,0) circle (0.05) node[anchor=south] (x_{2n}) {$x_{2n}$};
		\draw 		 (4,0) node[anchor=north] (-1) {$-1$};
		\draw 		 (8,0) node[anchor=north] (1) {$1$};
	\end{tikzpicture}
\end{center}

\begin{theorem}[Критерий Коши сходимости числовой последовательности]
	Числовая последовательность $\{x_n\}_{n=1}^{\infty}$ сходится тогда и только тогда, когда она фундаментальна.	
\end{theorem}
\begin{proof}
	Докажем необходимость.

	По условию $\{x_n\}_{n=1}^{\infty}$ сходится $\implies \exists \lim_{n \to \infty} x_n = a \in  \mathbb{R}$, тогда 
	\[
		\forall \varepsilon>0 \quad \exists N=N_(\varepsilon)\in \mathbb{N} \quad \forall n>N, \forall m>N : |x_{n} - a| <\frac{\varepsilon}{2}; |x_{m}-a|<\frac{\varepsilon}{2}
	.\]

	Рассмотрим $|x_{n}-x_{m} = | x_{n} - a + a - x_{m}| \le  |x_{n}-a| + |a-x_{m}| < \frac{\varepsilon}{2} + \frac{\varepsilon}{2} = \varepsilon$

	\[
	\forall  n>N, \forall m>N \quad (N=N(\varepsilon)\in \mathbb{N}) : |x_{n}-x_{m}|<\varepsilon \implies \{x_n\}_{n=1}^{\infty}
	.\] 
	Следовательно, $\{x_n\}_{n=1}^{\infty}$ фундаментальна.

	Докажем достаточность.

	По условию $\{x_n\}_{n=1}^{\infty}$ фундаментальна $\implies$ ограниченна $\implies \exists $ хотя бы одна предельная точка. Докажем, что эта предельная точка единственна ``от противного''. Предположим, что $\exists $ две предельные точки  последовательности $\{x_n\}_{n=1}^{\infty} : b$ и $b_1, b\neq b_1$. По определению предельной точки $\forall \varepsilon>0 \quad U_\varepsilon(b)$ и $U_\varepsilon(b_1)$  содержит бесконечно много членов последовательности $\{x_n\}_{n=1}^{\infty}$.

	$\{x_n\}_{n=1}^{\infty}$ фундаментальна $\implies \forall \varepsilon>0 \quad \exists N=N(\varepsilon)\in \mathbb{N} \quad \forall n,m>N : |x_n - x_{m}| < \varepsilon$.
\begin{center}
	\begin{tikzpicture}
		\draw[-{Classical TikZ Rightarrow[length=1mm]}] (0,0) -- (12,0);
		\draw[shift={(3.5,0)},color=black] (0pt,3pt) -- (0pt,-3pt) node[anchor=north] (b_1) {$b_1$};
		\draw[shift={(8.5,0)},color=black] (0pt,3pt) -- (0pt,-3pt) node[anchor=north] (b) {$b$};
		\draw (2,0) to[out=70,in=110] (5,0);
		\draw (7,0) to[out=70,in=110] (10,0);
		\draw            (4,1.5) node[anchor=north] (U_b) {$U_\varepsilon(b_1)$};
		\draw            (9,1.5) node[anchor=north] (U_a) {$U_\varepsilon(b)$};
		\draw (7,-1) node[anchor=north] (x_n) {$\{x_n\} $};
		\draw[->] (x_n) to (8,0.4);
		\draw[->] (x_n) to (4,0.4);
	\end{tikzpicture}
\end{center}

Т.к. $b_1 \neq  b \implies \varepsilon = \frac{|b_1-b}{6} > 0.$ Для выбранного $\varepsilon$ найдем соответствующий номер $N=N(\varepsilon)$
\[
\forall n, m >N : |x_{n} - x_{m}| < \frac{|b_1-b|}{6}
.\] 

Т.к. в $U_\varepsilon(b)$ и $U_\varepsilon(b_1)$ попадает бесконечно много элементов $\{x_n\}_{n=1}^{\infty}$, то
\[
	\exists  n_1>N, x_{n_1} \in  U_\varepsilon(b) \text{и} \exists m_1>N : x_{m_1} \in U_\varepsilon(b_1)
.\] 
\begin{multline}
	0 < |b-b_1| = |b-x_{n_1} + x_{n_1} - x_{m_1} + x_{m_1} - b_1| \le \\
	\le |x_{n_1} - b| + |x_{n_1} - x_{m_1}| + |x_{m_1} - b_1| <3\varepsilon = \\
	= \frac{3|b-b_1|}{6} = \frac{|b-b_1}{2}.
\end{multline}
 
\[
0 < |b-b_1| < \frac{|b-b_1|}{2}
.\] 
Противоречие, следовательно, $b=b_1$, а значит, $\{x_n\}_{n=1}^{\infty}$ имеет единственную предельную точку $\implies$ $\{x_n\}_{n=1}^{\infty}$ сходится (по теореме 4 о предельной точке последовательности).
\end{proof}

Пример: $\{x_n\}_{n=1}^{\infty}$, $x_{n} = 1 + \frac{1}{2} + \frac{1}{3} + \ldots + \frac{1}{n}$. Существует ли $\lim_{n \to \infty} x_n \in  \mathbb{R}$?

$\forall n \in  \mathbb{N} \quad m = 2n > n$
\[
	|x_{n} - x_{2n}| = |x_{2n} - x_{n}| = |1 + \frac{1}{2} + \ldots + \frac{1}{2n} - 1 - \frac{1}{2} -\ldots - \frac{1}{n}| =
\] 
\[
= \frac{1}{n+1} + \frac{1}{n+2} + .. + \frac{1}{2n} > \frac{1}{2n} + \frac{1}{2n} + \ldots + \frac{1}{2n} = \frac{1}{2}
.\] 
Следовательно, $\{x_n\}_{n=1}^{\infty}$ не является фундаментальной.
\[
	\exists \varepsilon=\frac{1}{2} \quad \forall N\in  \mathbb{N} \exists n>N \exists m=2n>N : |x_{n}-x_{2n}| > \frac{1}{2}
.\] 
Значит, не существует конечный $\lim_{n \to \infty} x_n$, т.е. последовательность не является сходящейся.


\begin{definition}
	Число $b$ или $+\infty(-\infty)$ называют частичным пределом числовой последовательности $\{x_n\}_{n=1}^{\infty}$ $\iff$
	\[
		\exists  \{x_n\}_{n=1}^{\infty} : \lim_{k \to \infty} x_{n_k} = b (+-\infty).
	.\] 
\end{definition}
	Если частичный предел есть конечное число, то это число является предельной точкой $\{x_n\}_{n=1}^{\infty}$.

	$(-1)^{n}$ - $+-1$ - частичные пределы.
	Наибольший частичный предел (может быть $+-\infty)$ называют верхним пределом числовой последовательности и обозначают $\overline{\lim_{n \to \infty} x_n}$.
	Наименьший частичный предел (может быть $+-\infty)$ называют нижним пределом числовой последовательности и обозначают $\underline{\lim_{n \to \infty} x_n}$.
	\[
		\underline{\lim_{n \to \infty} x_n} \le \overline{\lim_{n \to \infty} x_n}
	.\] 

Пример: $\{(-1)^{n}n\}_{n=1}^{\infty} = x_n$. $\underline{\lim_{n \to \infty} x_n} = +\infty$. $\underline{\lim_{n \to \infty} x_n} = -\infty$.

\begin{theorem}
	Последовательность $\{x_n\}_{n=1}^{\infty}$ сходится $\iff$
	\[
		\overline{\lim_{n \to \infty} x_n} = \underline{\lim_{n \to \infty} x_n} 
	.\] 
	и является конечным числом.
\end{theorem}




\newpage
\section{Пределы функций} 
\subsection{Определение предела по Коши}
Будем пользоваться следующими обозначениями:

$* \ : \quad a; \quad a+0; \quad a-0; \quad \infty; \quad +\infty; \quad -\infty$
 
$** : \quad b; \quad \infty; \quad +\infty; \quad -\infty$

\noindent Пусть функция $f(x)$ определена в некоторой проколотой окрестности $*$.
\begin{definition}[предела по Коши] $\lim_{x \to *} f(x) = ** \iff$
	\[
	\forall \varepsilon>0 \quad \exists \delta=\delta(\varepsilon)>0 \quad \forall x\in  \accentset{\circ}{U}_\delta(*)  \implies f(x) \in  U_\varepsilon(**)
	.\] 
\end{definition}

	\begin{tabular}{| c | l |}
		\hline
		$*$ & $x\in \accentset{\circ}{U}_\delta(*)$ \\
		\hline
		$a$     & $x \in \mathbb{R} : 0 < |x-a| < \delta$ \\
		\hline
		$a + 0$ & $x \in \mathbb{R} : a < x < a + \delta$ \\
		\hline
		$a - 0$ & $x \in \mathbb{R} : a-\delta < x < a $ \\
		\hline
		$\infty$ & $x \in \mathbb{R} : |x| > \delta$ \\
		\hline
		$+\infty$ & $x \in \mathbb{R} : x  > \delta$ \\
		\hline
		$-\infty$ & $x \in \mathbb{R} : x < -\delta$ \\
		\hline
	\end{tabular}
	\quad \quad
	\begin{tabular}{| c | l |}
			\hline
		$**$ & $f(x)\in U_\varepsilon(**)$ \\
			\hline
		$b$     & $|f(x)-b| < \varepsilon$ \\
			\hline
		$\infty$ & $|f(x)|>\varepsilon$ \\
			\hline
		$+\infty$ & $f(x)>\varepsilon$ \\
			\hline
		$-\infty$ & $f(x)<-\varepsilon$ \\
			\hline
	\end{tabular}
\\[12pt]
	\[
	x \in \accentset{\circ}{U}_\delta(a)
	\] 
	\begin{center}
		\begin{tikzpicture}
			\draw[-{Classical TikZ Rightarrow[length=1mm]}] (0,0) -- (12,0);
			\draw (3,0) to[out=70,in=110] (6,0);
			\draw (6,0) to[out=70,in=110] (9,0);
			\draw (6,0) node[anchor=north] (a) {$a$};
			\draw (3,0) node[anchor=north] (a-delta) {$a-\delta$};
			\draw (9,0) node[anchor=north] (a+d) {$a+\delta$};
			\draw[black, fill=white] (6,0) circle (0.07) node[anchor=north] (a) {$a$};
		\end{tikzpicture}
	\end{center}

	\[
	x \in \accentset{\circ}{U}_\delta(a+0)
	\] 
	\begin{center}
		\begin{tikzpicture}
			\draw[-{Classical TikZ Rightarrow[length=1mm]}] (0,0) -- (12,0);
			\draw (6,0) to[out=70,in=110] (9,0);
			\draw (6,0) node[anchor=north] (a) {$a$};
			\draw (9,0) node[anchor=north] (a+d) {$a+\delta$};
			\draw[black, fill=white] (6,0) circle (0.07) node[anchor=north] (a) {$a$};
		\end{tikzpicture}
	\end{center}

	\[
	x \in \accentset{\circ}{U}_\delta(a-0)
	\] 
	\begin{center}
		\begin{tikzpicture}
			\draw[-{Classical TikZ Rightarrow[length=1mm]}] (0,0) -- (12,0);
			\draw (3,0) to[out=70,in=110] (6,0);
			\draw (3,0) node[anchor=north] (a-delta) {$a-\delta$};
			\draw[black, fill=white] (6,0) circle (0.07) node[anchor=north] (a) {$a$};
		\end{tikzpicture}
	\end{center}

	\[
	x \in \accentset{\circ}{U}_\delta(\infty)
	\] 
	\begin{center}
		\begin{tikzpicture}
			\draw[-{Classical TikZ Rightarrow[length=1mm]}] (0,0) -- (12,0);
			\draw (0,1.7) to[out=0,in=110] (5,0);
			\draw (7,0) to[out=70,in=180] (12,1.7);
			\draw[black, fill=white] (5,0) circle (0.07) node[anchor=north] (-d) {$-\delta$};
			\draw[black, fill=white] (7,0) circle (0.07) node[anchor=north] (d) {$\delta$};
		\end{tikzpicture}
	\end{center}
	\[
	x \in \accentset{\circ}{U}_\delta(+\infty)
	\] 
	\begin{center}
		\begin{tikzpicture}
			\draw[-{Classical TikZ Rightarrow[length=1mm]}] (0,0) -- (12,0);
			\draw (7,0) to[out=70,in=180] (12,1.7);
			\draw[black, fill=white] (7,0) circle (0.07) node[anchor=north] (d) {$\delta$};
		\end{tikzpicture}
	\end{center}
	\[
	x \in \accentset{\circ}{U}_\delta(-\infty)
	\] 
	\begin{center}
		\begin{tikzpicture}
			\draw[-{Classical TikZ Rightarrow[length=1mm]}] (0,0) -- (12,0);
			\draw (0,1.7) to[out=0,in=110] (5,0);
			\draw[black, fill=white] (5,0) circle (0.07) node[anchor=north] (-d) {$-\delta$};
		\end{tikzpicture}
	\end{center}

	\begin{multline}
	\lim_{x \to a} f(x) = b \iff \\
	\iff \forall \varepsilon>0 \quad \exists \delta=\delta(\varepsilon)>0 \quad \forall x\in \mathbb{R}: 0< |x-a| < \delta \implies \\
	\implies |f(x)-b|<\varepsilon.
	\end{multline}

	\begin{center}
		\begin{tikzpicture}[domain=0:6]
		  %\draw[very thin,color=gray] (0,0) grid (8,8);

		  \draw[->] (0,0) -- (8,0) node[right] {$x$};
		  \draw[->] (0,0) -- (0,8) node[above] {$y$};
		  %\draw plot (\x,{0.03*exp(\x)}) node[right] {$f(x) = \frac{1}{20} \mathrm e^x$};
		  \draw[thick, color=red] plot (\x,{0.2*(\x)^2}) node[right] {$f(x)$}; %= \frac{x^2}{5}$};
		  %\draw[shift={(3.5,0)},color=black] (0pt,3pt) -- (0pt,-3pt) node[anchor=north] (a) {$a$};
		  \draw[dashed] (3.5,0) -- (3.5,2.45);
		  \draw[dashed] (3.5,2.45) -- (0,2.45) node[left] {$b$};
		  \draw[black, fill=white] (3.5,0) circle (0.07) node[anchor=north] (a) {$a$};

		  %\draw[shift={(5,0)},color=black] (0pt,3pt) -- (0pt,-3pt) node[anchor=north] (a+d) {$a+\delta$};
		  \draw[dashed] (5,0) -- (5,5);
		  \draw[dashed] (5,5) -- (0,5) node[left] {$b+\varepsilon$};;
		  \draw[black, fill=white] (5,0) circle (0.07) node[anchor=north] (a+d) {$a+\delta$};

		  %\draw[shift={(2,0)},color=black] (0pt,3pt) -- (0pt,-3pt) node[anchor=north] (a-d) {$a-\delta$};
		  \draw[dashed] (2,0) -- (2,0.82);
		  \draw[dashed] (2,0.82) -- (0,0.82) node[left] {$b-\varepsilon$};
		  \draw[black, fill=white] (2,0) circle (0.07) node[anchor=north] (a-d) {$a-\delta$};

	   	  \draw (3,0) node[anchor=north] {$x$};
		  \draw (3,0) -- (3,1.8) ;
		  \draw[->] (3,1.8) -- (0,1.8); 

		  \draw (0,0.82) to[out=130,in=230] (0,5);
		  \draw (-0.7,3) node[left] (U) {$U_\varepsilon(b)$};
		\end{tikzpicture}
	\end{center}
	$x \approx a$ с точностью $<\delta=\delta(\varepsilon) \implies f(x)\approx b$ с точностью $<\varepsilon$.

	\begin{multline}
		\lim_{x \to +\infty} f(x) = b \iff \\
		\iff \forall \varepsilon>0 \quad \exists \delta=\delta(\varepsilon) > 0 \quad \forall x \in \mathbb{R} : x > \delta \implies \\
		\implies |f(x) - b|< \varepsilon.
	\end{multline}
	Пример: $\lim_{x \to +\infty} \arctg{x} = \frac{\pi}{2}$.

	\begin{multline}
		\lim_{x \to -\infty} f(x) = b \iff \\
		\iff \forall \varepsilon > 0 \quad \exists \delta=\delta(\varepsilon) > 0 \quad \forall x\in \mathbb{R} : x < -\delta \implies \\
	\implies |f(x) - b| < \varepsilon.
	\end{multline}
	Пример: $\lim_{x \to -\infty} \arctg{x} = - \frac{\pi}{2}$.
\\[12pt]

	Если $* = a; \ \infty$, то $\lim_{x \to *} f(x)$ называется двусторонним пределом.
	Если $* = a+0; \ a-0;\ +\infty;\ -\infty$, то $\lim_{x \to *} f(x)$ называется односторонним пределом.
	Если $** = b$ (конечное число), то $\lim_{x \to *} f(x) = b$ называют конечным пределом.
	Если $** = \infty;\ +\infty;\ -\infty$, то $\lim_{x \to *} f(x)$ называют бесконечным.

\begin{theorem}[о связи двустороннего предела с односторонними]
	\[
		\exists \lim_{x \to a} f(x) = b \iff \exists \lim_{x \to a+0} f(x) = b \text{ и } \exists \lim_{x \to a-0} f(x) = b
	.\]
\end{theorem}
\begin{proof}
	Докажем необходимость. Распишем определение двустороннего предела по Коши.
	\begin{multline}
		\exists  \lim_{x \to a} f(x) = b \implies \\
		\implies \forall \varepsilon>0 \quad \exists \delta=\delta(\varepsilon) \quad \forall x\in \mathbb{R} : 0 < |x-a| < \delta \implies \\
		\implies |f(x) - b| < \varepsilon.
	\end{multline}	
	Рассмотрим неравенство $0 < |x-a| < \delta$.
	\begin{multline}
		0 < |x-a| < \delta \iff x \in (a-\delta, a) \cup (a; a + \delta) \implies \\
		\implies \begin{cases}
			\forall x \in  \mathbb{R} : a < x < a+\delta \implies |f(x)-b|<\varepsilon, \\
			\forall x \in  \mathbb{R} : a-\delta < x < a \implies |f(x)-b|<\varepsilon.
		\end{cases} \implies \\
		\implies \begin{cases}
			\exists \lim_{x \to a+0} f(x) = b, \\
			\exists \lim_{x \to a-0} f(x) = b.
		\end{cases}
	\end{multline}
	Докажем достаточность. Распишем определения односторонних пределов по Коши.
	\begin{multline}
		\lim_{x \to a+0} f(x) = b \iff \\
		\iff \forall \varepsilon>0 \quad \exists \delta_1 = \delta_1(\varepsilon)>0 \quad \forall x\in \mathbb{R} : a < x < a+\delta_1 \implies \\
		\implies |f(x) - b| < \varepsilon.
	\end{multline}
	\begin{multline}
		\lim_{x \to a-0} f(x) = b \iff \\
		\iff \forall \varepsilon>0 \quad \exists \delta_2 = \delta_2(\varepsilon)>0 \quad \forall x\in \mathbb{R} : a -\delta_2 < x < a \implies \\
		\implies |f(x) - b| < \varepsilon.
	\end{multline}
	Пусть $\delta = \min \{\delta_1, \delta_2\} > 0$. Тогда $\accentset{\circ}{U}_\delta(a)\subset (\accentset{\circ}{U}_{\delta_1}(a)\cap \accentset{\circ}{U}_{\delta_2}(a)) \implies$
	\[
		\implies (\forall x\in \mathbb{R} : 0 < |x-a| < \delta \implies |f(x)-b| < \varepsilon) \implies \exists \lim_{x \to a} f(x) = b
	.\]
\end{proof}

\begin{remark}[1]
		\[
		\lim_{x \to a} f(x) = \infty \iff \lim_{x \to a+0} f(x) = \infty, \lim_{x \to a-0} f(x) = \infty
		.\] 
\end{remark}
\begin{remark}[2]
	\[
		\lim_{x \to \infty} f(x) = \infty \iff \lim_{x \to +\infty} f(x) = b \ (\infty), \lim_{x \to -\infty} f(x) = b \ (\infty)
	.\] 
\end{remark}

\begin{definition}[Определение предела по Гейне]
	Пусть $f(x)$ определена в некоторой $\accentset{\circ}{U}(*)$.
	\[
	\lim_{x \to *} f(x) = ** \iff \forall \{x_n\}_{n=1}^{\infty} \subset \accentset{\circ}{U}(*) : \lim_{n \to \infty} x_n = * \implies \lim_{n \to \infty} f(x) = **,
	\]
где $x_n \neq * \ \forall n\in \mathbb{N}$.
\end{definition}

\begin{theorem}[об эквивалентности определений предела по Коши и Гейне]
	Определения предела по Коши и по Гейне эквивалентны.
\end{theorem}
Пример:
$\lim_{x \to 0} \sin{\frac{1}{x}}$ не определен.

\[
	x_n = \frac{1}{\pi n} \quad \lim_{n \to \infty} x_n = 0 \quad \lim_{n \to \infty} \sin{x_n} = \lim_{n \to \infty} \sin{\pi n} = 0
.\] 
\[
	y_n = \frac{1}{\frac{\pi}{2} + 2\pi n} \quad \lim_{n \to \infty} y_n = 0 \quad \lim_{n \to \infty} \sin{y_n} = \lim_{n \to \infty} \sin{\frac{\pi}{2} + 2\pi n} = 1
.\] 
$0 \neq 1 \implies \lim_{x \to 0} f(x)$ не существует.

\begin{theorem}[о единственности предела функции]
	Если существует $\lim_{x \to *} f(x) = b \in  \mathbb{R}$, то этот предел единственный (при $x \to *$).
\end{theorem}
\begin{proof}
	Воспользуемся определением предела по Гейне.
	\[
	\exists  \lim_{x \to *} f(x) = b \in \mathbb{R} \implies \forall \{x_n\}_{n=1}^{\infty}, x_n \neq  *, \lim_{n \to \infty} x_n = * \implies \lim_{n \to \infty} f(x_n) = b
	.\] 
	 Числовая последовательность $\{f(x_n)\}_{n=1}^{\infty}$ сходится, следовательно, имеет единственный предел $b$ (по теореме о единственности предела последовательности). 
\end{proof}

\begin{definition}
	Функция $f(x)$ называется локально ограниченной при  $x \to *$ (в точке $*$ или в окрестности $*$ ), если существуют такие $\accentset{\circ}{U}(*)$ и $M>0$, что $f(x)$ определена в  $\accentset{\circ}{U}(*) $ и $\forall x\in \accentset{\circ}{U}(*) : |f(x)| \le  M$. \\
Замечание: Если функция $f$ локально ограниченна при $x \to *$, то в точке $*$ такая функция может быть как определена, так и не определена.
\end{definition}


\begin{theorem}[о локальной ограниченности функции, имеющей конечный предел]
	Пусть $\exists \lim_{x \to *} f(x) = b \in \mathbb{R}$. Тогда $f(x)$ локально ограниченна при $x\to *$.
\end{theorem}
\begin{proof}
	По определению предела функции по Коши,
	\begin{multline}
	\lim_{x \to *} f(x)=b \iff \\
	\iff \forall \varepsilon> 0 \quad \exists \delta=\delta(\varepsilon)>0 \quad \forall x\in \accentset{\circ}{U}_\delta(*) \implies |f(x) - b| < \varepsilon \implies \\
	\implies \forall x\in \accentset{\circ}{U}_\delta(*) : |f(x)| = |f(x) -b + b| \le  |f(x) -b| + |b| < \varepsilon + |b| = M.
	\end{multline}
	Выберем любой $\varepsilon>0$, например, $\varepsilon=1$. Для соответствующей ему $\delta>0$ будет верно, что $\forall x\in \accentset{\circ}{U}_\delta(*) : |f(x)| < 1 + |b| = M$, а значит, $f(x)$ локально ограниченна при $x \to *$.
\end{proof}

\subsection{Бесконечно малые функции}
  \begin{definition}
  	Функцию $\alpha(x)$ называют бесконечно малой (б.м.) при $x \to  *$ тогда и только тогда, когда 
  	\[
  	  \lim_{x \to *} \alpha(x) = 0 \iff \forall  \varepsilon>0 \quad \exists \delta=\delta(\varepsilon) > 0 \quad \forall x \in  \accentset{\circ}{U}_\delta(*) \implies |\alpha(x)| < \varepsilon
  	.\] 
  \end{definition}
  \noindent Пример: Рассмотрим функцию $y = 2^{\frac{1}{x}}$. Если $x\to 0+0$, то
  \[
    \lim_{x \to 0+0} 2^{\frac{1}{x}} = [2^{+\infty}] = +\infty
  .\]
  Если же $x\to 0-0$, то
  \[
    \lim_{x \to 0-0} 2^{\frac{1}{x}} = [2^{-\infty}] = 0 \implies f(x) \text{ бесконечно малая при } x \to 0-0
  .\]
  
  \begin{theorem}[о связи функции, ее предела и бесконечно малой]
  \begin{multline}
  \lim_{x \to *} f(x) = b \iff \\
  \iff f(x) = b + \alpha(x) \text{, где }\alpha(x) \text{ --- бесконечная малая при }x\to *.
  \end{multline}
  \end{theorem}
  \begin{proof}
    Докажем необходимость.
    \begin{multline}
      \exists \lim_{x \to *} f(x) = b \iff \\
      \iff \forall \varepsilon>0 \quad \exists \delta=\delta(\varepsilon) \quad \forall x\in \accentset{\circ}{U}_\delta(*) : |f(x) - b| < \varepsilon.
    \end{multline}
    Положим $\alpha(x)=f(x)-b$, тогда $\forall x\in \accentset{\circ}{U}_\delta(*) : |\alpha(x)|<\varepsilon \implies$ \\
    $\implies \lim_{x \to *} \alpha(x) = 0 \implies \alpha(x)$ --- бесконечно малая при $x\to * \implies \\
    \implies f(x) = b + \alpha(x)$ при $x\to *$.
    
    Докажем достаточность. Пусть $f(x) = b + \alpha(x)$,  $\alpha(x)$ --- бесконечно малая при  $x \to  * \implies$, тогда $\alpha(x) = f(x) - b \to  0 \text{ при } x\to *$. По определению бесконечно малой,
    \begin{multline}
      \lim_{x \to *} \alpha(x) = 0 \iff \\
      \iff \forall  \varepsilon>0 \quad \exists \delta=\delta(\varepsilon) > 0 \quad \forall x \in \accentset{\circ}{U}_\delta(*) \implies |\alpha(x)| < \varepsilon \implies \\
      \implies \forall  \varepsilon>0 \quad \exists \delta=\delta(\varepsilon) > 0 \quad \forall x \in \accentset{\circ}{U}_\delta(*) \implies |f(x)-b| < \varepsilon \implies \\
      \implies \exists \lim_{x \to *} f(x) = b.
    \end{multline}
  \end{proof}

\subsection{Свойства бесконечно малых функций}
\begin{theorem}
	Пусть $\alpha(x)$ и  $\beta(x)$ - бесконечно малые при  $x\to *$. Тогда $\alpha(x) + \beta(x)$ --- бесконечно малые при  $x\to *$.
\end{theorem}
\begin{proof}
	$\alpha(x)$ и  $\beta(x)$ бесконечно малые при  $x\to *$ 
	\[
		\lim_{x \to *} \alpha(x) \implies \forall \varepsilon> 0 \quad \exists \delta_1=\delta_1(\varepsilon)>0 \quad \forall x\in \accentset {\circ}{U}_{\delta_1}(*) \implies |\alpha(x)| < \frac{\varepsilon}{2}	
	.\] 
	\[
		\lim_{x \to *} \alpha(x) \implies \forall \varepsilon> 0 \quad \exists \delta_2=\delta_2(\varepsilon)>0 \quad \forall x\in \accentset    {\circ}{U}_{\delta_2}(*) \implies |\beta(x)| < \frac{\varepsilon}{2}	
	.\] 

	Пусть $\delta = \min \{\delta_1, \delta_2\} $, если $*: a; a + 0; a - 0$ и  $\delta=\max \{\delta_1, \delta_2\} $, если $*: \infty;, +\infty, -\infty$.
	\[
		\implies \accentset{\circ}{U}_\delta(*) = \accentset{\circ}{U}_{\delta_1}(*) \cap \accentset{\circ}{U}_{\delta_2}(*) \implies
	.\] 
	\[
	\implies \forall 
	.\] 
	
\end{proof}

\begin{corollary}
Сумма конечного числа бесконечно малой при $x\to *$ есть бесконечно малая при $x \to  *$.
\end{corollary}

\begin{theorem}[произведение бесконечно малой на ограниченнную]
	Пусть $\alpha$ - бесконечно малая при $x \to *$, $f(x)$ локально ограниченна при  $x\to *$. Тогда $\alpha(x)\cdot f(x)$ есть бесконечно малая при $x\to *$.
\end{theorem}
	\begin{proof}
		$f(x)$ --- локально ограниченна при $x\to * \implies \exists \accentset{\circ}{U}_{\delta_1}(*) \quad \exists M>0 \quad \forall x \in  \accentset{\circ}{U}_{\delta_1}(*) : |f(x)| < M;$
		$\alpha(x)$ --- бесконечно малая при $x \to  * \implies \forall \varepsilon> 0 \quad \exists \delta_2=\delta_2(\varepsilon)>0 \quad \forall x\in \accentset    {\circ}{U}_{\delta_2}(*) \implies |\alpha(x)| < \frac{\varepsilon}{M} \implies$
		$\implies \accentset{\circ}{U}_\delta=\accentset{\circ}{U}_{\delta_1}(*)\cap \accentset{\circ}{U}_{\delta_2}(*)$

	\end{proof}
\[
	\lim_{x \to 0} x \sin{\frac{1}{x}} = 
.\] 
\[
	\lim_{x \to 0} x^2 \arctan{\frac{1}{x^{100}}} = 0
.\] 

\begin{theorem}[о произведении двух бесконечно малых]
	Пусть $\alpha(x)$, $\beta(x)$ --- бесконечно малые при $x \to *$. Тогда $\alpha(x)\beta(x)$ --- бесконечно малая при $x \to *$
\end{theorem}
\begin{proof}
	$\beta(x)$ --- бесконечно малая при $x \to  * \implies \lim_{x \to *} \beta(x) = 0 \implies $ по теореме о локальной ограниченной функции, имеющей конечный предел $\implies \beta(x)$ локально ограниченна при $x\to * \implies \alpha(x)\cdot \beta(x)$ --- произведение бесконечно малой на локально ограниченную при $x\to * \implies$ по теореме 2 $\alpha\cdot \beta$ --- бесконечно малые при.
\end{proof}

\begin{corollary}
	Произведение конечного числа бесконечно малых при $x \to  *$ есть бесконечно малая при $x \to *$.
\end{corollary}

\subsection{Арифметические операции с функциями, имеющими пределы}

\begin{theorem}
	Пусть $\exists \lim_{x \to *} f(x) = A \in  \mathbb{R}, \lim_{x \to *} g(x) = B\in  \mathbb{R}$
	Тогда
	\begin{enumerate}
		\item $\exists \lim_{x \to *} (f(x) \pm g(x)) = A\pm B$
		\item $\exists \lim_{x \to *} (f(x) g(x)) = AB$ 
		\item $B \neq 0 \implies \exists  \lim_{x \to *} \frac{f(x)}{g(x)} = \frac{A}{B}$
	\end{enumerate}
\end{theorem}
\begin{proof}
	$\exists  \lim_{x \to *} f(x) = A; \exists  \lim_{x \to *} g(x) = B \implies$ по теореме о связи функции, ее предела и бесконечно малой $\implies f(x) = A + \alpha(x), g(x) = B + \beta(x)$, где $\alpha(x), \beta(x)$ --- бесконечно малые при $x\to *$.
	
	\begin{enumerate}
		\item $f(x)\pm g(x) = (A + \alpha(x)) \pm (B+ \beta(x)) = A \pm B +\alpha(x) \pm \beta(x) = A \pm B + \gamma(x)$, где $\gamma(x)$ -- бесконечно малая при  $x \to * \implies$
		\item $f(x) \cdot  g(x) = (A + \alpha(x))(B+\beta(x))= AB+B\alpha(x)+A\beta(x) + \alpha(x)\beta(x) = AB + \gamma(x) \implies$ по теореме о связи предела функции, ее предела и бесконечно малой $\implies \lim_{x \to *} f(x)g(x) = AB$
		\item $\frac{f(x)}{g(x)} = \frac{A+\alpha(x)}{B+\beta(x)} = \frac{A}{B} + \frac{A+\alpha(x)}{B + \beta(x)} - AB$
			\[
			\gamma(x) = \frac{AB+B\alpha(x)-AB-A\beta(x)}{B(B+\beta(x)} = \frac{B\alpha(x)-A\beta(x)}{B} \cdot \frac{1}{B+\beta(x)}
			.\] 
			$\frac{B\alpha(x)-A\beta(x)}{B}$ --- бесконечно малая при $x \to  *$ по свойствам бесконечно малых.
			
			Докажем, что $\phi(x)= \frac{1}{B+\beta(x)}$ локально ограниченна при $x \to *$.
			$\beta(x)$ --- бесокнечно малая при $x \to * \implies \forall \varepsilon> 0 \quad \exists \delta=\delta(\varepsilon)>0 \quad \forall x\in \accentset    {\circ}{U}_\delta(*) \implies |\beta(x)| < \varepsilon$.
			Выберем $\varepsilon= \frac{|B|}{2}>0$, найдем соответствующее $\delta=\delta(\varepsilon)>0 \implies \forall  x \in  \accentset{\circ}{U}_\delta(*) : |\beta(x)| < \frac{|B|}{2}$ 
			\[
			\implies |B+\beta(x)| \ge  |B| - |\beta(x)| > |B| - \frac{|B|}{2} = \frac{|B|}{2} \implies
			.\] 
			\[
			\implies \forall x \in  \accentset{\circ}{U}_\delta(*) : |B + \beta(x)| > \frac{|B|}{2} >0 \implies
			.\] 
			\[
			\frac{1}{B+\beta(x)<\frac{2}{|B|}} \implies 
			.\]
			$\implies\phi(x) = \frac{1}{B+\beta(x)}$ локально ограниченна при  $x \to  *$
			$\gamma(x)$ --- произведение бесконечно малой на локально ограниченную при $x \to  *\implies$ по свойству бесконечно малой является бесконечно малой при $x \to *$.
	\end{enumerate}
\end{proof}

\begin{theorem}[о знакопостоянстве функции, имеющей ненулевой предел]
	Пусть $\exists  \lim_{x \to *} f(x) = b \neq 0$. Тогда $\exists \accentset{\circ}{U}_\delta(*) \quad \forall x \in \accentset{\circ}{U}_\delta(*) : |f(x)| > \frac{|b|}{2}$ 
	Кроме того, если $b>0$, то  $\forall x \in \accentset{\circ}{U}_\delta(*) : f(x) > \frac{b}{2} \implies f(x) > 0$, т.е. имеет тот же знак, что и предел;
	если $b<0$, то $\forall x \in  \accentset{\circ}{U}_\delta(*) : f(x) < \frac{b}{2} \implies f(x) <0$, т.е. имеет тот же знак, что и предел.
\end{theorem}

\begin{proof}
	\[
	\forall \varepsilon> 0 \quad \exists \delta=\delta(\varepsilon)>0 \quad \forall x\in \accentset    {\circ}{U}_\delta(*) \implies |f(x) - b| < \varepsilon
	.\] 
	Или $b - \varepsilon < f(x) < b + \varepsilon$.
	Выберем $\varepsilon = \frac{|b|}{2} > 0$, найдем соответствующий $\delta = \delta(\varepsilon) > 0 \implies$
	\[
	\forall x \in  \accentset{\circ}{U}_\delta(*) : |f(x)| = |b + f(x) - b| \ge  |b| - |f(x) - b| > |b| - \frac{|b|}{2}
	.\] 
	\[
	\implies |f(x)| > \frac{|b|}{2}
	.\] 
Пусть $b > 0$, тогда $\varepsilon = \frac{|b|}{2} = \frac{b}{2} > 0 \implies \forall x \in  \accentset{\circ}{U}_\delta(*) \quad f(x) > b - \varepsilon = b - \frac{b}{2} = \frac{b}{2} > 0$
\end{proof}


\end{document} % конец документа
