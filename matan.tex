\documentclass[a4paper,12pt]{article} % добавить leqno в [] для нумерации слева

%%% Работа с русским языком
\usepackage{cmap}					% поиск в PDF
\usepackage{mathtext} 				% русские буквы в фомулах
\usepackage[T2A]{fontenc}			% кодировка
\usepackage[utf8]{inputenc}			% кодировка исходного текста
\usepackage[english,russian]{babel}	% локализация и переносы

%%% Дополнительная работа с математикой
\usepackage{amsmath,amsfonts,amssymb,amsthm,mathtools} % AMS
\usepackage{icomma} % "Умная" запятая: $0,2$ --- число, $0, 2$ --- перечисление

%% Номера формул
\mathtoolsset{showonlyrefs=true} % Показывать номера только у тех формул, на которые есть \eqref{} в тексте.

%% Шрифты
\usepackage{euscript}	 % Шрифт Евклид
\usepackage{mathrsfs} % Красивый матшрифт

%% Свои команды
\DeclareMathOperator{\sgn}{\mathop{sgn}}
% Теоремы
\newtheorem{definition}{Определение}
\newtheorem{theorem}{Теорема}
\newtheorem{lemma}{Утверждение}
%\newtheoremstyle{named}{}{}{\itshape}{}{\bfseries}{.}{.5em}{\thmnote{#3's }#1}
%\theoremstyle{named}
%\newtheorem*{namedtheorem}{Theorem}

\usepackage{textcomp}

% figure support
\usepackage{import}
\usepackage{pdfpages}
\usepackage{transparent}
\usepackage{xcolor}

\newcommand{\incfig}[2][1]{%
    \def\svgwidth{#1\columnwidth}
    \import{./figures/}{#2.pdf_tex}
}

\pdfsuppresswarningpagegroup=1



%% Перенос знаков в формулах (по Львовскому)
\newcommand*{\hm}[1]{#1\nobreak\discretionary{}
{\hbox{$\mathsurround=0pt #1$}}{}}

%%% Заголовок
\author{Власова Елена Александровна}
\title{Лекции по математическому анализу для 1 курса ФН2, 3}
\date{2024-2025 год.}

\begin{document} % конец преамбулы, начало документа

\maketitle

\newpage
\tableofcontents
\newpage

\part*{Элементарные функции и их пределы}

\section{Введение}
\subsection{Элементы теории множеств}
\subsection{Кванторные операции}
\subsection{Метод математической индукции}

\section{Множество действительных чисел}
\subsection{Аксиоматика действительных чисел}
\begin{definition}
	Множество $\mathbb{R}$ называется множеством действительных чисел, если элементы этого множества удовлетворяют следующему комплексу условий:
\begin{enumerate}
	\item На множестве $\mathbb{R}$ определена операция сложения ``+'', то есть задано отображение, которое каждой упорядоченной паре $(x, y)\in \mathbb{R}^2$ ставит в соответствие элемент из $\mathbb{R}$, называемый суммой $x+y$ и удовлетворяющий следующим аксиомам:
		\begin{enumerate}
			\item $\exists 0\in \mathbb{R}$, такой, что $\forall x\in \mathbb{R}:x+0=0+x=x$;
			\item $\forall x \ \exists$ противоположный элемент ``$-x$'', такой, что $x+(-x)=(-x)+x=0$;
			\item Ассоциативность. $\forall x, y, z\in \mathbb{R}:(x+y)+z=x+(y+z)$;
			\item Коммутативность. $\forall x, y\in \mathbb{R}:x+y=y+x$.
		\end{enumerate}
	\item На $\mathbb{R}$ определена операция умножения ``$\cdot$'', то есть $\forall(x, y)\in\mathbb{R}^2$ ставится в соответствие элемент $(x\cdot y)\in \mathbb{R}$.
		\begin{enumerate}
			\item $\exists $ нейтральный элемент $1\in \mathbb{R}$, такой, что $\forall x\in \mathbb{R}:1\cdot x=x\cdot 1=x$;
			\item $\forall x\in \mathbb{R}\backslash\{0\} \ \exists $ обратный элемент ``$x^{-1}$'', такой, что $x\cdot x^{-1}=x^{-1}\cdot x=1$;
			\item Ассоциативность. $\forall x, y, z\in \mathbb{R}\backslash\{0\}:(x\cdot y)\cdot z=x\cdot (y\cdot z)$;
			\item Коммутативность. $\forall x, y\in \mathbb{R}\backslash\{0\}:x\cdot y=y\cdot x$.
		\end{enumerate}
\end{enumerate}
	Операция умножения дистрибутивна по отношению к сложению. 
		$$\forall x, y, z\in \mathbb{R}:(x+y)z=xz+yz$$
\begin{enumerate}
\setcounter{enumi}{2}
	\item Отношения порядка. Для $\mathbb{R}$ определено отношение ``$\le $ ''.
		\begin{enumerate}
			\item $\forall x\in \mathbb{R}:x\le x$;
			\item $\forall x, y\in \mathbb{R}:(x\le y \land y\le x)\implies x=y$;
				
				\ldots
		\end{enumerate}
\end{enumerate}
	
\end{definition}

\subsection{Геометрическая интерпретация $\mathbb{R}$}
\subsection{Числовые промежутки}
\subsection{Бесконечные числовые промежутки}
\subsection{Окрестности точки}
\subsection{Принцип вложенных отрезков (Коши-Кантора)}
\begin{definition}
	Пусть $\{x_n\}_{n=1}^{\infty}$ --- последовательность некоторых множеств. Если $\forall n\in \mathbb{N}:X_n\supset X_{n+1}$, то эта последовательность называется последовательностью вложенных отрезков.
\end{definition}

\subsection{Ограниченные и неограниченные числовые множества}
\subsection{Точные грани числового множества}
\subsection{Принцип Архимеда}


\section{Функции или отображения}
\subsection{Понятие функции}
\subsection{Ограниченные и неограниченные числовые множества}
\subsection{Обратные функции}
\subsection{Чётные и нечётные функции}
\subsection{Периодические функции}
\subsection{Сложная функция (композиция)}
\subsection{Основные элементарные функции}

\section{Числовые последовательности и их пределы}
\begin{definition}
	$f:\mathbb{N}\rightarrow \mathbb{R}$ --- числовая последовательность, т.е. $\{x_n\}_{n=1}^\infty$, $x_n \in \mathbb{R}$.
\end{definition}

\subsection{Ограниченные и неограниченные числовые последовательности}

\begin{definition}
	Числовая последовательность $\{x_n\}_{n=1}^\infty$ называется 

	\begin{enumerate}
		\item ограниченной сверху, если $\exists M\in \mathbb{R}: \forall n\in \mathbb{N}: \ x_n\le M$;
		\item ограниченной снизу, если \space $\exists M\in \mathbb{R}: \forall n\in \mathbb{N}: \ x_n\ge M$;
		\item ограниченной, если \space\space\space\space\space\space\space\space\space $\exists M\in \mathbb{R}: \forall n\in \mathbb{N}:|x_n| \le M$;
		\item неограниченной, если \space\space\space\space\space\space $\exists M\in \mathbb{R}: \forall n\in \mathbb{N}:|x_n|> M$;
	\end{enumerate}
\end{definition}

\subsection{Предел числовой последовательности}
\begin{definition}
	Число $a\in\mathbb{R}$ называется пределом числовой последовательности, если $\forall \varepsilon>0$ существует такой номер $n$, зависящий от $\varepsilon$, что $\forall$ натурального числа $N>n$ верно неравенство $|x_n - a| < \varepsilon$.
	$$\lim_{n\to\infty} x_n = a$$
\end{definition}

\subsection{Бесконечные пределы}
\subsection{Свойства сходящихся последовательностей}
\subsection{Монотонные числовые последовательности}
\begin{definition}
	Числовая последовательность $\{x_n\}_{n=1}^{\infty}$ называется 
	\begin{enumerate}
		\item возрастающей, если \ \ $\forall n\in \mathbb{N}:x_{n}<x_{n+1}$;
		\item убывающей, если \ \ \ \ \ \ \ $\forall n\in \mathbb{N}:x_{n}>x_{n+1}$;
		\item неубывающей, если \ \ \ \ $\forall n\in \mathbb{N}:x_{n}\le  x_{n+1}$;
		\item невозрастающей, если $\forall n\in \mathbb{N}:x_{n}\ge x_{n+1}$
	\end{enumerate}
\end{definition}
Для монотонных числовых последовательностей ограниченность является достаточным условием для сходимости.
\begin{theorem}[Вейерштрасса о сходимости монотонных числовых последовательностей]
	Если последовательность не убывает и ограничена сверху, то она является сходящейся.
	Если последовательность не возрастает и ограничена снизу, то она является сходящейся. В общем, любая монотонная последовательность сходится.
\end{theorem}
\begin{proof}
	Пусть $\{x_n\}_{n=1}^{\infty}$ не убывает и ограничена сверху $\implies\exists M\in \mathbb{R} :\forall n\in \mathbb{N}:x_{n}\le M\implies$ множество значений этой последовательности $\{x_1, x_2, x_3, \ldots, x_{n}, \ldots\}=A$ является ограниченным сверху числовым множеством $\implies\exists \sup A\in \{x_n\}_{n=1}^{\infty}=a$, то есть
	\begin{enumerate}
		\item $\forall n\in \mathbb{N}:x_{n}\le a$;
		\item $\forall \varepsilon>0 \ \exists N=N(\varepsilon)\in \mathbb{N}:x_{N}>a-\varepsilon$.
	\end{enumerate}
Т.к. $\{x_n\}_{n=1}^{\infty}$ --- неубывающая последовательность $\implies$

\begin{multline}
\implies \forall n>N=N(\varepsilon):x_{n}\ge x_N \implies \\
\implies a-\varepsilon<x_N\le x_{n}\le a<a+\varepsilon\implies \\
\implies a-\varepsilon<x_{n}<a+\varepsilon \implies |x_{n}-a|<\varepsilon \implies \\
\implies \forall \varepsilon>0 \quad \exists N=N(\varepsilon)\in \mathbb{N}:\forall n>N:|x_{n}-a|<\varepsilon \implies \\
\implies \exists \lim_{n \to \infty}x_{n} = a\in \mathbb{R} \implies \{x_n\}_{n=1}^{\infty} \text{ сходится.}
\end{multline}
Если $\{x_n\}_{n=1}^{\infty}$ -- невозрастающая и ограниченная снизу последовательность, то $\exists \lim_{n \to \infty} x_{n}=\inf A, A = \{x_1, x_2, \ldots, x_{n}, \ldots\}$. Доказательство аналогично.
\end{proof}

\subsection{Число $e$}
\subsection{Гиперболические функции}
\subsection{Предельные точки числового множества}
\subsection{Предельные точки числовых последовательностей}

\end{document} % конец документа
